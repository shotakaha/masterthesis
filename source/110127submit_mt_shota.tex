\documentclass[11pt]{jreport}
%\documentclass[11pt, draft]{jreport}

\usepackage{geometry}	% See geometry.pdf to learn the layout options. There are lots.
\geometry{a4paper}    % ... or a4paper or a5paper or ...

%余白の設定
\setlength{\textwidth}{40zw}
%\setlength{\textheight}{44\baselineskip}
\setlength{\textheight}{40\baselineskip}
\addtolength{\textheight}{\topskip}
%\setlength{\hoffset}{0.46cm}
%\addtolength{\textheight}{2.4cm}

\usepackage[dvipdfmx]{graphicx, color}
\usepackage{wrapfig}
\usepackage{mediabb}
\usepackage{subfigure}
\DeclareGraphicsRule{.tif}{png}{.png}{`convert #1 `dirname #1`/`basename #1 .tif`.png}
\usepackage{wallpaper}
\usepackage{utf}
\usepackage{ulem}
%\usepackage{umoline}
%\usepackage{epstopdf}
\usepackage{float}
\usepackage{lineno}
%\linenumbers
\usepackage{amsmath, amssymb}
\usepackage{type1cm}

\西暦

\bibliographystyle{jplain}

\usepackage[dvipdfm,%
bookmarks=false,%
bookmarksnumbered=true,%
bookmarkstype=toc,%
colorlinks=true,%
linkcolor=black,%
citecolor=black,%
pdftitle={Mizucheの開発},
pdfauthor={Shota TAKAHASHI},
pdfkeywords={T2K, Mizuche, neutrino, Kyoto University}]{hyperref}

% bookmark(しおり)の文字化け対策
\ifnum 42146=\euc"A4A2
\AtBeginDvi{\special{pdf:tounicode EUC-UCS2}}\else
\AtBeginDvi{\special{pdf:tounicode 90ms-RKSJ-UCS2}}\fi

%%%%%%%%%%%%%%%%%%%%%%%%%%%%%%%%%%%%%%%%%%%%%%%%%%%%%%%%%%%%%%%%%%%%%%%%%%%%%%%
% ===== particles =====
\newcommand{\pizero}{$\pi^{0}$}

% ===== neutrino interaction =====
\newcommand{\ccqe}{$\nu + p \rightarrow \mu + n$}
\newcommand{\ncqe}{$\nu + N \rightarrow \nu + N$}
\newcommand{\ccp}{$\nu + p \rightarrow \mu + n + \pi^{0}$}
%\newcommand{\nc1p}
%\newcommand{\dis}

% = math sequences =
% \bra and \ket must be used in math mode
\newcommand{\bra}[1]{\langle #1 |}
\newcommand{\ket}[1]{ | #1 \rangle}

% = units =
\newcommand{\cmcm}{cm$^{2}$}

% = Tips =
\newcommand{\figref}[1]{図\ref{#1}}
\newcommand{\tabref}[1]{表\ref{#1}}
\newcommand{\equref}[1]{式(\ref{#1})}
\newcommand{\secref}[1]{第\ref{#1}章}
\renewcommand{\bibname}{参考文献}

% = Mizuche Original=
\newcommand{\fv}{\mathrm{FV}}
\newcommand{\ov}{\mathrm{OV}}
\newcommand{\ww}{\mathrm{(w/ FVwater)}}
\newcommand{\wow}{\mathrm{(w/o FVwater)}}
\newcommand{\photon}{\mathrm{photon}}
\newcommand{\pe}{\mathrm{p.e.}}
\newcommand{\nd}{\mathrm{ND}}
\newcommand{\sk}{\mathrm{SK}}
\newcommand{\miz}{\mathrm{Miz}}

% = color =
\newcommand{\red}[1]{\textcolor{red}{\textbf{#1}}}
\newcommand{\comment}[1]{\red{#1}\footnote{\red{#1}}}
\newcommand{\memo}[1]{\footnote{\red{#1}}}

% = subfigure
\renewcommand*{\thesubfigure}{(\arabic{subfigure})}

% = subsubsubsection =
%\setcounter{secnumdepth}{6}
%\makeatletter
%\newcommand{\subsubsubsection}{\@startsection{paragraph}{4}{\z@}%
%  {1.5\Cvs \@plus.5\Cdp \@minus.2\Cdp}%
%  {.5\Cvs \@plus.3\Cdp}%
%  {\reset@font\normalsize\sffamily}
%}

%%%%%%%%%%%%%%%%%%%%%%%%%%%%%%%%%%%%%%%%%%%%%%%%%%%%%%%%%%%%%%%%%%%%%%%%%%%%%%%
\begin{document}


%\ThisCenterWallPaper{.5}{fig/06_1.jpg}
\begin{titlepage}

\title{修士論文\\ニュートリノ反応数測定のための\\小型水チェレンコフ検出器"Mizuche"の開発}
\author{京都大学大学院理学研究科 物理学・宇宙物理学専攻\\
物理学第二教室 高エネルギー物理学研究室\\
\UTF{9AD9}橋 将太}
\date{\today}

%\ThisULCornerWallPaper{.3}{fig/shota-pooh_christmas.pdf}
%\begin{figure}[!h]
%\centering
%\includegraphics[width=0.2\textwidth]{fig/06_1.jpg}
%\end{figure}

\end{titlepage}

\maketitle

\pagenumbering{roman}
%%%%%%%%%%%%%%%%%%%%%%%%%%%%%%%%%%%%%%%%%%%%%%%%%%%%%%%%%%%%%%%%%%%%%%%%%%%%%%%
\begin{abstract}
\end{abstract}

\tableofcontents
%\listoffigures
%\listoftables



%%%%%%%%%%%%%%%%%%%%%%%%%%%%%%%%%%%%%%%%%%%%%%%%%%%%%%%%%%%%%%%%%%%%%%%%%%%%%%%
\chapter{はじめに}
\pagenumbering{arabic}
\pagestyle{bothstyle}
%%%%%%%%%% %%%%%%%%%% %%%%%%%%%% %%%%%%%%%% %%%%%%%%%% %%%%%%%%%%





%%%%%%%%%%%%%%%%%%%%%%%%%%%%%%%%%%%%%%%%%%%%%%%%%%%%%%%%%%%%%%%%%%%%%%%%%%%%%%%%
%%%%%%%%%%%%%%%%%%%%%%%%%%%%%%%%%%%%%%%%%%%%%%%%%%%%%%%%%%%%%%%%%%%%%%%%%%%%%%%%
\if0
\chapter{データ収集システム:MizuDAQ}

\section{概要}
\section{フロントエンド部}
\subsection{ATM}
\subsection{GONG}
\section{リアエンド部}
\subsection{SMP}
\section{トリガー部}
\subsection{TRG}
\fi

%%%%%%%%%%%%%%%%%%%%%%%%%%%%%%%%%%%%%%%%%%%%%%%%%%%%%%%%%%%%%%%%%%%%%%%%%%%%%%%%
%%%%%%%%%%%%%%%%%%%%%%%%%%%%%%%%%%%%%%%%%%%%%%%%%%%%%%%%%%%%%%%%%%%%%%%%%%%%%%%%
%\chapter{ニュートリノビーム測定}

%%%%%%%%%%%%%%%%%%%%%%%%%%%%%%%%%%%%%%%%%%%%%%%%%%%%%%%%%%%%%%%%%%%%%%%%%%%%%%%%
%%%%%%%%%%%%%%%%%%%%%%%%%%%%%%%%%%%%%%%%%%%%%%%%%%%%%%%%%%%%%%%%%%%%%%%%%%%%%%%%
\chapter{まとめ}

本研究では、T2K実験前置検出器ホールにて開発中の小型水チェレンコフ検出器``Mizuche''の開発を行った。検出器シミュレーションによる期待される性能評価と、強度解析・耐震解析の結果を踏まえた実機の構造の決定ならびに製作、そして、使用する光電子増倍管の相対的量子効率・電流増幅率の測定を行った。

MIzuche検出器は、水とニュートリノの反応によって生じる荷電粒子(主にミューオン)が放出するチェレンコフ光を、周りに設置した光電子増倍管で検出することによって、ニュートリノを観測する水チェレンコフ光検出器である。

T2K実験の後置検出器であるスーパーカミオカンデと同じニュートリノ反応標的(水)と検出原理(チェレンコフ光)を持つ本検出器により振動前のニュートリノ反応数を測定することにより、スーパーカミオカンデへの外挿を系統誤差を小さく抑えて行うことそ最終目標としている。

本検出器は有効体積0.5トンの内タンク(FV)と一回り大きい外タンク(OV)を同軸上に配置した2層構造をしている。外タンクと内タンクの間には300 mmのバッファー層を設け、FVの端で起こったニュートリノ反応によるミューオンでも十分なチェレンコフ光を発生させることができるようになっている。FVとOVは物理的に区切られており、その内部の水はそれぞれ独立して充填することが可能である。そのため、FVの水だけを抜き差しして測定を行うことが可能である。

本検出器の測定原理は、FV水ありと、FV水なしの2状態で測定を行い、その残差をFVで起きたニュートリノ反応数として計数することである。
これは2状態の差をとることにより、OVでの反応は相殺し、FVでの反応だけが残るからである。だたし、この測定原理が成り立つためには、2状態のOVでのニュートリノ反応検出効率が一致している必要がある。

そこで、ニュートリノ反応に対する検出器シミュレーションを行い、2状態のOVの検出効率をそれぞれ見積もった。
その結果、期待される総光量に対するカットを150 p.e.に設定すると、2状態のOVの検出効率はよく一致し、測定原理が成り立つことを示すことができた。またその際に、シグナルに対するOV混入イベントの割合を3\%程度にまで抑えて測定することが可能だということが分かった。

次に強度・耐震性の確認を行いながら、検出器の詳細設計を行った。強度・耐震性の解析にはANSYSという強度解析ツールを使用した。材料の引張強度に対して安全係数を3に設定した設計を行い、業者に製作を依頼した。検出器を満水試験による変形量は、ANSYSで得られた結果とほぼ同じ結果を示した。強度の安全性は問題ないと判断し、検出器を前置検出器ホール地下2階へインストールした。

光電子増倍管のキャリブレーションに関しては、必要数164本に対して、現在までに153本の相対的量子効率、電流増幅率曲線の測定を終えた。本測定のセットアップの再現性による光量補正を行った結果、相対的量子効率は13\%程度のばらつきがあることが分かった。今後は、合わせて測定した電流増幅率曲線を利用して、電流増幅率の制御を行い、ある入射光量に対して全ての光電子増倍管からの出力が一様となるよう印加電圧の調整を行う予定である。


今後の予定として、まず、検出器への光電子増倍管取り付けやケーブリングなどのアセンブリー作業を完了させる。次に、LED光源を用いた検出器応答のキャリブレーションや、ニュートリノビーム由来の壁からのミューオンを用いた光量キャリブレーションなどの作業を行う。そして、FV水ありの状態でニュートリノ反応数の測定を開始し、十分なデータを取得できた後は、水なしの状態での測定を開始する。そして、T2K実験の大強度ビームに対する小型水チェレンコフ検出器の実用性の検証、および前置検出器部分でのニュートリノ反応数測定精度2\%を目指したい。


%%%%%%%%%%%%%%%%%%%%%%%%%%%%%%%%%%%%%%%%%%%%%%%%%%%%
%%%%%%%%%%%%%%%%%%%%%%%%%%%%%%%%%%%%%%%%%%%%%%%%%%%%
\chapter*{謝辞}

修士課程の2年間は、とても短く感じられましたが、多くの方々に支えられて、充実した日々を送ることができました。ここに感謝の意を表したいと思います。本当にありがとうございます。

中家剛教授、市川温子准教授、そして小林隆教授には、T2K実験という世界最先端の実験場で本研究の機会を与えてくださったことに感謝いたします。

特に、市川温子准教授には、のんびりな私を要所要所で引き締めていただき、根気よく丁寧にここまでご指導いただいたことを深く感謝しております。

坂下健助教には、本研究を進める上でとてもたくさんのご助言をいただきました。
また、研究者としての姿勢を間近で学ばせていただきました。
村上明さんには、検出器シミュレーションをはじめ、本研究の至る所でたくさんお世話になりました。
山内隆寛くんに、年末の光電子増倍管の測定を手伝っていただいておかげで、測定をほぼ終わらせることができました。本当にありがとう。

検出器の作製に関して、スズノ技研株式会社の皆様に本当に感謝しております。できあがった検出器を初めて見たときの嬉しさはいまでも忘れません。本当にありがとうございます。

検出器のインストールや地下での作業には第一鉄工株式会社の皆様に大変お世話になりました。
いろいろと急なお願いが多いにも関わらず、引き受けてくださり本当にありがとうございます。

メカサポートの田井野ご夫妻と石井さんには、光電子増倍管の準備およびコネクタ付けの際には大変お世話になりました。本当にありがとうございます。

%この研究をみんなと一緒にできて本当に良かったと思います。

J-PARCでの生活では、大谷将士さん、木河達也くん、鈴木研人くん、矢野孝臣さん、とたまに来る松村知恵さんのおかげで、楽しく過ごせました。大谷さんは、何かと研究の進捗を気にかけて下さり、アドバイスなどもたくさんしていただきました。ありがとうございます。\newline

最後に、これまで研究生活を支えてくれた家族と最愛の人に心から感謝します。本当にありがとう。\newline
\begin{flushright}
2011年1月 吉日\\
\UTF{9AD9}橋 将太
\end{flushright}



\input{bibbib}
%\bibliography{mt_ref}

\end{document}
