%%%%%%%%%%%%%%%%%%%%%%%%%%%%%%
% 非推奨なパッケージ
%
% 2011年の執筆時点で利用したパッケージで、
% 現在(2024年)には非推奨になっているパッケージの一覧です。
% 非推奨になった理由と、代替パッケージ/後継パッケージも併記しました。
%%%%%%%%%%%%%%%%%%%%%%%%%%%%%%
% subfigureは、現在非推奨なパッケージです。
% 代わりにsubcaptionを利用します。
%\usepackage{subfigure}
%
% ulemは、下線や取り消し線を引くためのパッケージです。
% umolineは、ulemの代替パッケージですが、こちらも非推奨です。
% 代わりにsoulやsoulutf8を利用します。
%\usepackage{ulem}
%\usepackage{umoline}
%
% type1cmは、フォントのスケーリングを行うためのパッケージです。
% luatexja-fontspecパッケージでフォントを設定できるため、不要です。
% \usepackage{type1cm}
%
% mediabbは、PDのバウンディングボックス情報を取得するためのパッケージです。
% 現在のgraphicxには、この機能が統合されているため、不要です。
%\usepackage{mediabb}
%
% utfは、UTF-8エンコーディングを扱うためのパッケージです。
% luatexja-fontspecパッケージでUTF-8エンコーディングを扱うため、不要です。
%\usepackage{utf}
%
% epstopdfは、EPSファイルをPDFファイルに変換するためのパッケージです。
% 現在のgraphicxには、この機能が統合されているため、不要です。
% またモダンLaTeXでは、PDFファイルを直接取り込むことが推奨されています。
%\usepackage{epstopdf}