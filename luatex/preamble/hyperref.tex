%%%%%%%%%%%%%%%%%%%%%%%%%%%%%%
% ハイパーリンク関係
%%%%%%%%%%%%%%%%%%%%%%%%%%%%%%
% hyperrefパッケージを読み込むと、目次や参考文献などのリンクを自動的に作成できます。
% \hypersetup コマンドで、リンクの色や枠線の設定、PDFのメタデータの設定などを行います。
% このパッケージは、他のパッケージよりも後に読み込む必要があります。
\usepackage{hyperref}
\hypersetup{
    %%%%%%%%%%%%%%%%%%%%
    % PDFのしおりの設定
    %%%%%%%%%%%%%%%%%%%%
    % Unicode対応を有効にする
    unicode=true,    % [false, true] (default: false)
    % しおりを作成する
    % bookmarks=true,    % [true, false] (default: true)
    % しおりのスタイル
    % bookmarkstype=toc, % [toc, number, none] (default: toc)
    % しおりにセクション番号を表示する
    bookmarksnumbered=true,  % [false, true] (default: false)
    %%%%%%%%%%%%%%%%%%%%
    % リンクの設定
    %%%%%%%%%%%%%%%%%%%%
    % 長いリンクを改行する
    breaklinks=true,    % [false, true] (default: false)
    % リンクを見えないようにする(色や枠線を非表示)
    % hidelinks,    % [false, true] (default: false)
    % リンクの枠線の設定
    pdfborder={0 0 0},    % [0 0 0] (default: [0 0 1])
    % PDF内のリンクをカラー表示する
    colorlinks=true,    % [false, true] (default: false)
    % URLの色
    urlcolor=black,    % [red, blue, ...] (default: cyan)
    % ファイルリンク(file://)の色
    filecolor=black,    % [red, blue, ...] (default: magenta)
    % メニューリンクの色
    menucolor=black,    % [red, blue, ...] (default: red)
    % ハイパーリンクの色
    linkcolor=black,    % [red, blue, ...] (default: red)
    % 文献引用の色
    citecolor=black,    % [red, blue, ...] (default: green)
    %%%%%%%%%%%%%%%%%%%%
    % メタデータの設定
    %%%%%%%%%%%%%%%%%%%%    
    % PDFのタイトル
    pdftitle={Mizucheの開発},    % 文字 (default: [])
    % PDFのサブタイトル
    pdfsubject={修士論文},    % 文字 (default: [])
    % PDFの著者
    pdfauthor={Shota TAKAHASHI},    % 文字 (default: [])
    % PDFを作成したツール
    %pdfcreator={LuaTeX},    % 文字 (default: LuaTeX)
    % PDFを作成した組版システム
    %pdfproducer={LuaTeX},    % 文字 (default: LuaTeX)
    % PDFのキーワード
    pdfkeywords={T2K, Mizuche, neutrino, Kyoto University},    % 文字 (default: [])
    %%%%%%%%%%%%%%%%%%%%
    % PDFの表示設定
    %%%%%%%%%%%%%%%%%%%%
    % PDFを開いた時の表示モード
    % pdfstartview={FitH},    % [Fit, FitH, FitV, FitB, FitBH, FitBV] (default: Fit)
    % PDFを開いた時のナビゲーションパネルの表示モード
    pdfpagemode={UseOutlines},    % [UseNone, UseThumbs, UseOutlines, FullScreen] (default: UseNone)
    % PDFを開いた時にウィンドウサイズに合わせる
    pdffitwindow=true,    % boolean: [false, true] (default: false)
}

\renewcommand{\sectionautorefname}{章}
\renewcommand{\figureautorefname}{図}
\renewcommand{\tableautorefname}{表}

% cleverefは相互参照をサポートするパッケージです。
% \crefコマンドで、参照先のカウンターに応じた接頭辞を出力します
% \crefnameコマンドで、カウンター名を変更できます。
\usepackage{cleveref}
% \crefname{環境名}{単数}{複数形}
\crefname{section}{セクション}{セクション}
\crefname{figure}{図}{図}
\crefname{table}{表}{表}