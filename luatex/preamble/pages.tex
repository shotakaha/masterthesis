%%%%%%%%%%%%%%%%%%%%%%%%%%%%%%
% ページ設定
% geometryはページ設定できるパッケージです。
% ドキュメントクラスによって、ページ設定のデフォルト値は異なりますが、
% このパッケージを利用することで、より詳細な設定が可能です。
% 例えば、用紙サイズ、余白の設定、本文エリアの中央揃えなどを指定できます。
% レイアウトを表示するためにshowframeオプションを指定します。
% 余白の設定は、marginオプションで指定します。
%%%%%%%%%%%%%%%%%%%%%%%%%%%%%%
\usepackage{geometry}
\geometry{
    % レイアウトを表示
    showframe,    % [true, false] (default: false)
    % 用紙サイズ
    a4paper,    % a4paper, a5paper, b5paper, letterpaper, legalpaper, executivepaper (default: a4paper)
    % 余白の設定
    margin=25mm,    % [length] (default: 1.5in)
    %left=30mm,    % [length] (default: 1.5in)
    %right=30mm,    % [length] (default: 1.5in)
    %top=30mm,    % [length] (default: 1.5in)
    %bottom=30mm,    % [length] (default: 1.5in)
    % 余白の自動調整
    %heightrounded,    % [true, false] (default: false)
    % 本文エリアを中央揃え
    %centering,    % [true, false] (default: false)
}

%余白の設定
%\setlength{\textwidth}{40\zw}
%\setlength{\textheight}{44\baselineskip}
%\setlength{\textheight}{40\baselineskip}
%\addtolength{\textheight}{\topskip}
%\setlength{\hoffset}{0.46cm}
%\addtolength{\textheight}{2.4cm}
% fancyhdrはヘッダー・フッターを設定するためのパッケージです。
\usepackage{fancyhdr}
\pagestyle{fancy}

%%%%%%%%%%%%%%%%%%%%%%%%%%%%%%
% 箇条書き関係
%%%%%%%%%%%%%%%%%%%%%%%%%%%%%%
\usepackage{enumitem}

% linenoは行番号を表示するためのパッケージです。
% 草稿を共有する際に有効にすると、行番号を使って修正の指摘を受けやすくなります。
% \usepackage{lineno}
%\linenumbers