\begin{abstract}
T2K(Tokai-to-Kamioka)長基線ニュートリノ実験は、
茨城県東海村にある大強度陽子加速器施設J-PARCの陽子ビームを使って生成した人工ニュートリノビームを、
295km離れた岐阜県飛騨市にあるスーパーカミオカンデに向けて飛ばし、
その飛行中に起こるニュートリノ振動を観測する実験である。

2009年4月に稼働開始したこの実験は、ミューオンニュートリノ消失モードによる振動パラメータの精密測定、
および未発見の電子ニュートリノ出現モードの世界初観測を目標にしている。

ニュートリノ振動の精密測定には、ニュートリノビームのフラックス、
ニュートリノ反応断面積、検出効率の不定性に起因する系統誤差を低く抑えることが必要となってくる。
そこで、前置検出器部分において、後置検出器であるスーパーカミオカンデと同じ測定原理・ニュートリノ反応標的を持つ
水チェレンコフ光検出器で測定することができれば、これらの系統誤差を削減することが期待できる。

T2K実験のニュートリノビーム強度では、大型の水チェレンコフ光検出器に対するニュートリノ反応レートが大きく、
1バンチ内で多数のニュートリノが反応してしまい、バンチ毎にイベントを区別して測定することが困難になる。
そのため、本実験では容積を\num{2.5}トンと小型化し、ニュートリノ反応数を計数することに特化した検出器の開発を行うことにした。

本検出器は、直径\qty{1400}{\mm}、長さ\qty{1600}{\mm}のタンクの周囲に光電子増倍管を\num{164}本配置し、
水とニュートリノ反応により生じた荷電粒子のチェレンコフ光を検出する。
検出器内部に有効体積領域として、直径\qty{800}{\mm}、長さ\qty{1000}{\mm}のアクリル製内タンクが入っており、
物理的に区切られた2層構造になっている。

本実験は次の2つを目標にしている。

\begin{enumerate}
    \item 前置検出器部分で水チェレンコフ光を用いたニュートリノ反応数の測定(目標精度2\%)
    \item スーパーカミオカンデでのニュートリノ反応予測数の精度向上
\end{enumerate}

まず、大強度ニュートリノビームに対しても水チェレンコフ検出器が有効であることを実証しながら、
2\%の精度での測定を目指す。その結果を基に、我々は本検出器をT2K前置検出器群と合わせて利用し、
T2K実験の測定感度の向上を最終目標として目指す。

本論文では、T2K実験の前置検出器ホールにて開発を行った新型の水チェレンコフ検出器について、
その実験原理、検出器シミュレーションによる期待される性能の評価、
強度解析・耐震解析の結果を踏まえた構造体の設計、
使用する光電子増倍管のキャリブレーションについて報告する。

\end{abstract}