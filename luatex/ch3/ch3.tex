%%%%%%%%%%%%%%%%%%%%%%%%%%%%%%%%%%%%%%%%%%%%%%%%%%%%%%%%%%%%%%%%%%%%%%%%%%%%%%%%
%%%%%%%%%%%%%%%%%%%%%%%%%%%%%%%%%%%%%%%%%%%%%%%%%%%%%%%%%%%%%%%%%%%%%%%%%%%%%%%%
\chapter{小型水チェレンコフ検出器 Mizuche}

%%%%%%%%%% %%%%%%%%%% %%%%%%%%%% %%%%%%%%%% %%%%%%%%%% %%%%%%%%%%
\section{Mizucheの概要}
Mizuche実験とは、J-PARC加速器によって生成された直後のニュートリノビーム中のニュートリノの個数を、後置検出器であるスーパーカミオカンデと同じ水チェレンコフ光検出器で測定する実験である。

ニュートリノ振動測定の精密測定には、ニュートリノビームのフラックス、ニュートリノと反応標的の反応断面積、検出器の検出効率の不定性に起因する系統誤差を低く抑えることが必要となってくる。そこで、生成直後のニュートリノビームの性質を、前置検出器で測定し、スーパーカミオカンデに外挿することにより、これらの系統誤差を小さく抑えることができる。特に、スーパーカミオカンデと同じ測定原理・検出装置を持つ水チェレンコフ光検出器で測定することにより、これらの系統誤差を削減することができる。

実際に、過去のK2K実験では1キロトンの水チェレンコフ光検出器を使用することで、系統誤差をキャンセルした測定に大きく貢献している。しかし、T2K実験の場合は、ニュートリノビーム強度がK2K実験よりも2桁強いため、ニュートリノ反応レートが大きくなり、1キロトンもの大容積では、1バンチ内で多数のニュートリノが反応してしまい、バンチ毎にイベントを区別して測定することが困難になってしまう。そこで、本実験では容積を2.5トンと小型化し、ニュートリノ反応数を数えることに特化した検出器の開発を行うことにした。

\section{Mizucheの目的}

本実験は後置検出器であるスーパーカミオカンデと同じタイプの水チェレンコフ検出器を用いて前置検出器部分でのニュートリノ反応数測定を行い、外挿することで、系統誤差を低く抑えたニュートリノ反応数予測を目指す実験である。
それに向けて、本実験では次の2つを目標にしている。
\begin{enumerate}
\item 前置検出器部分で水チェレンコフ光検出器を用いたニュートリノ反応数測定\\(目標精度2\%)
\item スーパーカミオカンデでのニュートリノ反応数予測の精度向上
\end{enumerate}

第一目標として、前置検出器部分でのニュートリノ反応数測定の精度を2\%で行うことを目指すことを掲げている。まずここまでで、T2K実験の大強度ニュートリノビームに対しても水チェレンコフ検出器が有効であることを実証する。
次の目標としては、ここまでに得られた結果を元に、本検出器をT2K前置検出器群と合わせて利用し、
スーパーカミオカンデでのニュートリノ反応数予測を行う。ここで、\figref{MizuSKFlux}に本検出器とスーパーカミオカンデで予測されるニュートリノフラックスを示した。このように、ほぼ同じ形のフラックス、同じニュートリノ反応標的(水)、同じ検出原理(チェレンコフ光)を用いることで、最終的には、系統誤差を抑えた外挿を行うことにより、T2K実験の測定感度の向上に貢献したいと考えている。

\begin{figure}[htbp]
  \begin{minipage}{0.47\textwidth}
    \subfigure[Mizuche]{
\includegraphics[bb=128 475 450 708, width=1\textwidth]{fig/MCNeutrinoFlux.pdf}
   \label{MizuFlux}}
  \end{minipage}
  \hfill
  \begin{minipage}{0.47\textwidth}
    \subfigure[スーパーカミオカンデ]{
\includegraphics[bb=255 191 822 575, width=1\textwidth]{fig/MizucheSKFlux3.pdf}
   \label{SKFlux}}
  \end{minipage}
    \caption[Mizucheとスーパーカミオカンデでのニュートリノフラックス]{Mizucheとスーパーカミオカンデでのニュートリノフラックス}
  \label{MizuSKFlux}
\end{figure}



%\subsubsection{振動解析と系統誤差}
%以下に、とある振動解析の手法と、Mizucheを使用した場合に、どのような系統誤差を抑えることができるのかを示す。
%\begin{equation}
%N_{SK}^{exp}  =  R_{Far/Near} \times N_{Miz}^{obs}
%\label{Extrapolation}
%\end{equation}
%ここで、
%\begin{equation}
%R_{Far/Near} = \frac{N_{SK}^{MC}}{N_{Miz}^{MC}} = \frac{\int \Phi_{SK}^{MC} \times \sigma_{SK} \times \epsilon_{SK}}{\int \Phi_{Miz}^{MC} \times \sigma_{Miz} \times \epsilon_{Miz}}
%\label{Extrapolation2}
%\end{equation}

%\begin{itemize}
%\item $N_{SK}^{exp} \cdots $ SKでのニュートリノ反応の予測数
%\item $N_{Miz}^{obs} \cdots $ Mizucheでの実際のニュートリノ反応観測数
%\item $R_{Far/Near} \cdots$ Far - Near比
%\item $N_{SK, Miz}^{MC} \cdots$ MCのよるSK, Mizucheでのニュートリノ反応数
%\item $\Phi_{SK, Miz}^{MC} \cdots$ MCのよるSK, Mizucheでのエネルギースペクトル
%\item $\sigma_{SK, Miz} \cdots$ SK, Mizucheでのニュートリノ反応断面積
%\item $\epsilon_{SK, Miz} \cdots$ SK, Mizucheでの検出効率
%\end{itemize}

%%%%%%%%%% %%%%%%%%%% %%%%%%%%%% %%%%%%%%%% %%%%%%%%%% %%%%%%%%%%
\section{Mizucheの実験原理}

本検出器は、水中を高速で走る荷電粒子が放出するチェレンコフ光をとらえることにより粒子を検出する、スーパーカミオカンデと同じ水チェレンコフ光検出器である。

ニュートリノが水中の水素原子核や酸素原子核と反応し荷電粒子が生成される。その時の荷電粒子(主にミューオン)が水中を進むことによって放出されるチェレンコフ光を、検出器の周りに配置した光電子増倍管で観測する。
%\comment{反応の絵}


本検出器は\figref{TankConcept}のような、外タンク(直径1400 mm、長さ1600 mm)の内側に、一回り小さな内タンク(直径800 mm、長さ1000 mm)を抱えた、2層構造をしている。内タンクの容積約0.5 m$^{3}$(= 500 kg)を有効体積(fiducial volume: FV)と定義する。

FV内でのニュートリノ反応数は、(1) FV内に水がある状態と、(2) FV内に水がない状態の2状態で測定を行い、その残差から求める。この測定原理の詳細については後述する。

\begin{figure}[htb]
\centering
%\includegraphics[bb=0 0 575 320, scale=0.5]{fig/MizucheTankConcept.pdf}
\includegraphics[bb=0 0 1012 578, width=1\textwidth]{fig/MizucheTankConcept2.pdf}
\caption[Mizuche検出器の概念設計図]{Mizuche検出器の概念設計図。青:内タンク($\phi$800 mm$\times$1000 mm);水:外タンク($\phi$1400 mm$\times$1600mm);桃:3in. 光電子増倍管$\times$164本。FVの端でのニュートリノ反応によるチェレンコフ光を観測できるよう、外タンクと内タンクの間には300mmの領域(Outer Volume: OV)を設定した。}
\label{TankConcept}
\end{figure}


\subsection{チェレンコフ放射}
チェレンコフ放射とは、荷電粒子が媒質中を運動する時、その速度が媒質中の光速度よりも速い場合に光を放射する現象である。1934年にP. A. チェレンコフによって発見されたことからその名が付いている。

\subsubsection{チェレンコフ角とエネルギー閾値}
媒質の屈折率を$n$、荷電粒子の進行方向とチェレンコフ光の放出方向のなす角度を$\theta_{c}$とすると、$\theta_{c}$は荷電粒子の速度$\beta c$によって決まり、以下の関係が成り立つ。%(\equref{CherenkovAngle})
\begin{equation}
\cos \theta_{c} = \frac{1}{n\beta}
\label{CherenkovAngle}
\end{equation}

チェレンコフ光は、荷電粒子の進行方向を軸とする円錐面に沿って放出される。荷電粒子のエネルギーが十分大きく、その速度が光速に近い速度($\beta =1$)であるとき、チェレンコフ角$\theta_{c}$は最大となる。また、エネルギーが小さくなるにつれ、チェレンコフ角$\theta_{c}$は狭くなり、エネルギーが低すぎるとチェレンコフ光は放出されない。チェレンコフ光が放出される最低速度$\beta_{t}$(threshold velocity)と、そのときのエネルギー閾値$E_{t}$(energy threshold)は次式で表すことができる(\equref{ThresholdVelocity}、\equref{EnergyThreshold})。

\begin{equation}
\beta_{t} = \frac{1}{n}
\label{ThresholdVelocity}
\end{equation}

%\begin{eqnarray}
%\frac{p_{t}}{E_{t}} & = & \frac{1}{n}\\
%\frac{\sqrt{E_{t}^{2}-m^{2}}}{E_{t}}  & = & \frac{1}{n} \\
%n^{2} (E_{t}^{2}-m^{2}) & = & E_{t}^{2} \\
%n^{2}E_{t}^{2} - n^{2}m^{2} & = & E_{t}^{2} \\
%(n^{2}-1)E_{t}^{2} & = & n^{2}m^{2} \\
%E_{t}^{2} & = & \frac{n^{2}m^{2}}{n^{2}-1}\\
%
%\end{eqnarray}

\begin{equation}
E_{t} = \frac{nm}{\sqrt{n^{2}-1}}
\label{EnergyThreshold}
\end{equation}

%\begin{eqnarray}
%p_{t} & = & \sqrt{E_{t}^{2}-m^{2}}\\
%& = & \sqrt{\frac{n^{2}m^{2}}{n^{2}-1}-m^{2}}\\
%& = & \sqrt{\frac{n^{2}m^{2}-m^{2}(n^{2}-1)}{n^{2}-1}}\\
%& = & \sqrt{\frac{m^{2}}{n^{2}-1}}\\
%& = & \frac{m}{\sqrt{n^{2}-1}} \ (= \beta_{t} E_{t})
%\end{eqnarray}

水の場合、屈折率$n\sim1.33$なので、最大チェレンコフ角$\theta_{c} \sim 42^{\circ}$、$\beta_{t} \sim 0.75$となる。
また、\tabref{ThresholdByParticle}に主な粒子別のエネルギーと運動量の閾値をまとめた。


\begin{table}[htbp]
\caption[主な粒子の水に対するチェレンコフ光放出のエネルギー閾値と運動量閾値]{主な粒子の水に対するチェレンコフ光放出のエネルギー閾値$E_{t}$と運動量閾値$p_{t}$}
\begin{center}
\begin{tabular}{c|ccc}
\hline \hline
& 静止質量 $m$ [MeV/c$^{2}$] & $E_{t}$ [MeV] & $p_{t}$ [MeV/c]\\
 \hline
e$^{\pm}$	& 0.511	& 0.775 & 0.583\\
$\mu^{\pm}$	& 105.7 & 160.3 & 120.5\\
$\pi^{\pm}$	& 139.6 & 211.7 & 159.2 \\
p$^{+}$	& 938.2	& 1423 & 1070\\
\hline \hline
\end{tabular}
\end{center}
\label{ThresholdByParticle}
\end{table}%

\subsubsection{単位長さあたりに放出されるチェレンコフ光子数}
荷電粒子の電荷が$ze$ [C]であるとき、単位飛程、単位波長あたりに放出される光子数$N_{\photon}$は次のように表すことができる(\equref{dNdXdL})。
%\begin{eqnarray}
%\frac{d^{2}N_{photon}}{dxd\lambda} & = & \frac{2 \pi \alpha z^{2}}{\lambda^{2}} \left( 1 - \frac{1}{\beta^{2}n^{2}(\lambda)}\right) \\
%& = & \frac{2 \pi \alpha z^{2}}{\lambda^{2}} \sin^{2} \theta_{c}
%\end{eqnarray}

\begin{equation}
\frac{d^{2}N_{\photon}}{dxd\lambda} =  \frac{2 \pi \alpha z^{2}}{\lambda^{2}} \sin^{2} \theta_{c}
\label{dNdXdL}
\end{equation}
ここで、$\lambda$はチェレンコフ光の波長、$\alpha \simeq 1/137$は微細構造定数である。
これを波長で積分すると、次式が得られる(\equref{dNdX})。

\begin{equation}
\frac{dN_{\photon}}{dx} =  2 \pi \alpha z^{2} \sin^{2} \theta_{c} \left( \frac{1}{\lambda_{1}}-\frac{1}{\lambda_{2}} \right) \ , (\lambda_{1} < \lambda_{2})
\label{dNdX}
\end{equation}

これまでの式から荷電粒子がエネルギーを失うに従って、チェレンコフ角$\theta_{c}$が小さくなると共に、チェレンコフ光の強度も減少していくことが分かる。

典型的な光電子増倍管で検出可能な波長は300 nm $\sim$ 650 nmである。この範囲を考慮すると、運動量 1 GeV/c の荷電粒子が単位長さ進むときに放出する光子数は$N_{\photon}/dx \sim 823\sin^{2}\theta_{c}\ $ [photon/cm]程度と見積もることができる。
ミューオン、電子のそれぞれに対して、運動量とチェレンコフ角の関係と、荷電粒子が単位長さ進むあたりに放出されるチェレンコフ光子数の関係をそれぞれ\figref{MizucheCheDeg}と\figref{MizuchedPhoton}に図示した。

また、本検出器表面積の約6\%が光電子増倍管で覆われていることと、光電子増倍管の量子効率が約20\%であると仮定して、荷電粒子が単位長さ進んだときに期待される光量(光電子数)を見積もったものを\figref{MizuchedPE}に図示する。これから1 cmあたり$3\sim4$ photon しか放出されないことが分かる。そのため、内タンク(FV)の外側に300 cmのバッファー層(OV)を設けることで、FVで生じた荷電粒子が、検出するのに十分な光量のチェレンコフ光を放射することを保証した。30 cm進んだ際に検出できる光量を\figref{MizuchePE}に示す。これらの検出器設計詳細については後述する。

\begin{figure}[htbp]
\begin{minipage}{0.47\textwidth}
\centering
\includegraphics[bb=0 0 500 484, width=1\textwidth]{fig/MizucheCheDeg.pdf}
\caption[ミューオンと電子の運動量とチェレンコフ角の関係]{ミューオンと電子の運動量とチェレンコフ角の関係。黒線はミューオン、赤線は電子を表す。}
\label{MizucheCheDeg}
\end{minipage}
%\end{figure}
\hfil
%\begin{figure}[htbp]
\begin{minipage}{0.47\textwidth}
\centering
\includegraphics[bb=0 0 500 484, width=1\textwidth]{fig/MizuchedPhoton.pdf}
\caption[ミューオンと電子の運動量と単位飛程あたりに放出されるチェレンコフ光子数の関係]{ミューオンと電子の運動量と単位飛程あたりに放出されるチェレンコフ光子数の関係。黒線はミューオン、赤線は電子を表す。}
\label{MizuchedPhoton}
\end{minipage}
\end{figure}


\begin{figure}[htbp]
\begin{minipage}{0.47\textwidth}
\centering
\includegraphics[bb=0 0 500 484, width=1\textwidth]{fig/MizuchedPE.pdf}
\caption[Mizucheで検出できる単位飛程あたりの光電子数]{Mizucheで検出できる単位飛程あたり光電子数。光電被覆率 6.24\%、量子効率19\%とした。黒線はミューオン、赤線は電子を表す。}
\label{MizuchedPE}
\end{minipage}
%\end{figure}
\hfil
%\begin{figure}[htbp]
\begin{minipage}{0.47\textwidth}
\centering
\includegraphics[bb=0 0 500 484, width=1\textwidth]{fig/MizuchePE.pdf}
\caption[荷電粒子が 300 mm 進むときに期待される光電子数数]{荷電粒子が 300 mm 進むときに期待される光電子数。黒線はミューオン、赤線は電子を表す。}
\label{MizuchePE}
\end{minipage}
\end{figure}

\newpage
\subsection{測定原理}
前述したように、本検出器は、次の2状態で測定を行い、その残差を求める。
\begin{enumerate}
\item FV内に水がある状態(FV水あり)
\item FV内に水がない状態(FV水なし)
\end{enumerate}

この測定原理について、\figref{EventCategory}を用いて詳しく説明する。
上段の図は(1) FV内に水がある状態での測定、下段の図は(2) FV内に水がない状態での測定を表している。それぞれの場合でチェレンコフ光が発生する要因によって4つに場合分けして図示した。

\begin{figure}[htbp]
\centering
\includegraphics[bb=16 103 971 616, width=1\textwidth]{fig/MizucheEventCategory.pdf}
\caption[測定原理の概略]{測定原理の概略。上段:FV水ありの測定、下段:FV水なしの測定。チェレンコフ光が発生する要因によって4つに場合分けした。}
\label{EventCategory}
\end{figure}

\subsubsection{1. FV内で起こるニュートリノ反応(左端の図)}
本検出器のシグナルイベントである。FV内でのニュートリノ反応は、FV水ありの状態でしか起こらないため、その残差はFV内での反応数、すなわちシグナルイベントとなる。


\subsubsection{2. FV外で起こるニュートリノ反応(左から2番目の図)}
FV外(OV)でのニュートリノ反応は、FV水あり、FV水なしとも起こるため、両状態の検出効率が全く等しい場合、差をとれば反応数は相殺する。相殺しなかった場合は、バックグラウンドとなる。それぞれの場合で期待される検出効率については、\secref{MonteCalro}の検出器シミュレーションにて詳述する。

\subsubsection{3. 砂ミューオンによるチェレンコフ放射(左から3番目の図)}
砂ミューオンが発生するチェレンコフ光によるイベントである。砂ミューオンとは、前置検出器ホールの壁とニュートリノが反応したことにより生じたミューオンのことである。このイベントはFV水あり、水なしでも起こるため、OVで起こるニュートリノ反応同様、差をとれば反応数は相殺する。相殺しなかった場合はバックグラウンドとなる。

\subsubsection{4. 検出器外からの中性粒子による反応(右端の図)}
砂ミューオンの発生同様、前置検出器ホールの壁とニュートリノが反応したことによる中性粒子(主に中性子)が、検出器内の水と反応し、荷電粒子を生成するイベントである。
FV内でこの反応が生じた場合、差をとるとバックグラウンドとして残ることになる(OVで生じた反応は相殺する)。中性子による反応がどの程度起きるかは検出器シミュレーションにより見積もる。

%%%%%%%%%%%%%%%%%%%%%%%%%%%%%%%%%%%%%
%%%%%%%%%%%%%%%%%%%%%%%%%%%%%%%%%%%%%
%%%%%%%%%%%%%%%%%%%%%%%%%%%%%%%%%%%%%
\section{簡単なニュートリノ反応数の見積もり}

\subsection{有効体積内でのニュートリノ反応頻度}
T2Kニュートリノビームのデザイン強度で、FV内でのニュートリノ反応数を見積もった。ニュートリノビームのフラックスを$\Phi_{\nu}$、反応断面積を$\sigma_{\nu}$、標的粒子数を$n$とすると、ニュートリノ反応数$N_{\nu}$は次の式で表すことができる。
\begin{equation}
N_{\nu}  = \Phi_{\nu} \times \sigma_{\nu} \times n
\end{equation}

陽子ビーム強度750 kWの時のニュートリノフラックス$\Phi_{\nu}=1.85 \times 10^{6}\ \mathrm{[/cm^{2}/sec]}$、$\ \sigma_{\nu}=0.63 \times 10^{-38}\ \mathrm{[cm^{2}/nucleon]}$、水500 kg中の核子数$n=3.01\times10^{29}\ \mathrm{[nucleon]}$より、$3.5\times10^{-3}\ \mathrm{[events/sec]}$が期待されると見積もった。\tabref{EventRateEstimation}に計算に用いた条件をまとめた。

\begin{table}[htbp]
\caption{ニュートリノ反応数の見積もりに使用した条件}
\begin{center}
\begin{tabular}{ccl}
\hline \hline
陽子ビームエネルギー & 30 & [GeV]\\
%陽子ビーム数 & $3.3 \times 10^{14}$ & [/\cmcm/spill]\\
陽子ビーム強度 & 0.75 &[MW] \\
%ビーム周期(スピル間隔) & 2.11 & [sec] \\
%1スピル当たりの陽子数 & $3.3 \times 10^{14}$ & [protons/spill]\\
%ニュートリノビームフラックス & $6.5 \times 10^{6}$ & [/\cmcm /spill] \\
ニュートリノビームフラックス & $1.85 \times 10^{6}$ & [/\cmcm /sec] \\
ニュートリノエネルギー & 0.7 & [GeV]\\
水反応断面積 & $0.63 \times 10^{-38}$ & [\cmcm/nucleon]\\
FV質量 & 0.5 & [ton] \\
核子数 & $3.01\times10^{29}$ & [nucleon]\\
\hline
ニュートリノ反応頻度 & $3.50 \times 10^{-3}$ & [events/sec]\\
\hline \hline
\end{tabular}
\end{center}
\label{EventRateEstimation}
\end{table}%

\subsection{測定時間による統計誤差}
限られたビームタイムの中で、FV内に水がある状態とない状態の2状態の測定を行わなければならない。そこで、それぞれの測定時間による統計誤差が最小となるよう、最適な水あり/水なしの測定時間の比を見積もった。

ニュートリノフラックスを$\Phi_{\nu}$、反応断面積を$\sigma_{\nu}$、アボガドロ数を$N_{A}$とし、FV水ありの時の体積を$V_{1}$・測定時間を$T_{1}$・ニュートリノ反応数を$N_{1}$、FV水なしでの体積を$V_{2}$・測定時間を$T_{2}$・ニュートリノ反応数を$N_{2}$とすると、
それぞれの状態でのニュートリノ反応数は次のようになる。

\begin{description}
\item [FV水あり]%
\begin{equation}
%N_{1}=\sigma_{\nu}F_{\nu}n_{1}T_{1}=\sigma_{\nu}F_{\nu}N_{A}V_{1}T_{1}
N_{1} = \sigma_{\nu}\Phi_{\nu}N_{A}V_{1}T_{1}
\label{Nww}
\end{equation}
%
\item [FV水なし]
\begin{equation}
%N_{2}=\sigma_{\nu}F_{\nu}n_{2}T_{2}=\sigma_{\nu}F_{\nu}N_{A}V_{2}T_{2}
N_{2} = \sigma_{\nu}\Phi_{\nu}N_{A}V_{2}T_{2}
\label{Nwow}
\end{equation}
\end{description}

ここで、測定時間の比を$T_{1}:T_{2}=a:b$とすると、FV内でのニュートリノ反応数は次のようになる(測定時間をFV水ありに合わせた)。

\begin{equation}
N_{\fv} = N_{1}-\frac{T_{1}}{T_{2}}N_{2} = N_{1}-\frac{a}{b}N_{2}
\label{Nfv}
\end{equation}

$N_{1}$、$N_{2}$はポアソン過程だと仮定すると、それぞれの統計誤差は次のようになる。
\begin{eqnarray}
\sigma_{N_{1}} & = & \sqrt{N_{1}} \label{sigma1}\\
\sigma_{N_{2}} & = & \sqrt{N_{2}} \label{sigma2}
\end{eqnarray}

また、誤差の伝播式より、$N_{\fv}$の統計誤差は次のようになる。
\begin{equation}
\sigma_{N_{\fv}} = \sqrt{\sigma_{N_{1}}^{2}+\left(\frac{a}{b}\right)^{2}\sigma_{N_{2}}^{2}} \label{sigmafv}
\end{equation}

全体に対する統計誤差の割合を計算すると、
\begin{eqnarray}
\frac{\sigma_{N_{\fv}}}{N_{\fv}} & = & \frac{\sqrt{\sigma_{N_{1}}^{2}+\left(\frac{a}{b}\right)^{2}\sigma_{N_{2}}^{2}}}{N_{1}-\frac{a}{b}N_{2}}\\
%
& = & \frac{\sqrt{N_{1} + \left(\frac{a}{b}\right)^{2}N_{2}}}{N_{1}-\frac{a}{b}N_{2}}\\
& = & \frac{1}{\sqrt{N_{1}}} \cdot%
\frac{1}{\sqrt{V_{1}}} \cdot%
\frac{\sqrt{V_{1}+\frac{a}{b}V_{2}}} {V_{1}-V_{2}}\\
\end{eqnarray}

ここで、$\sqrt{V_{1}+\frac{a}{b}V_{2}}$が最小となる$a$、$b$を考える。
\\
\\
相加平均・相乗平均の定理より、
\begin{eqnarray}
A + B & \ge & 2\sqrt{AB} \ \text{(等号成立は$A=B$)} \label{sksj}\\
V_{1} + \frac{a}{b}V_{2} & \ge & 2\sqrt{V_{1}\frac{a}{b}V_{2}}\\
\text{等号成立は} & & V_{1} = \frac{a}{b}V_{2} \Rightarrow a:b=V_{1}:V_{2}\\
\text{(このとき} & & V_{1}+\frac{a}{b}V_{2} = 2V_{1}\text{\ )}
\end{eqnarray}

したがって、
\begin{equation}
T_{1}:T_{2} = V_{1}:V_{2}
\end{equation}
測定時間による統計誤差を最小にするためには、測定時間を測定状態の体積比に配分すれば良いことが分かった。

\subsection{1年間で期待されるニュートリノ反応数}
1年のビームタイムを100日と仮定し、前節のとおりに測定時間を配分したときに期待されるニュートリノ反応数を見積もった。FV水ありの体積は2.5トン、FV水なしの体積は2.0トンであるので、測定時間はそれぞれ56日と44日になる。ビーム強度を100 kW(750 kW)と仮定したときのイベント数を\tabref{EventEstimationYear}にまとめた。FV内でのニュートリノ反応は1日あたり41(304)イベントが期待できる。


\begin{table}[htbp]
\caption[期待されるニュートリノ反応数]{期待されるニュートリノ反応数}
\begin{center}
\begin{tabular}{cccc}
\hline \hline
測定状態 & 測定日数 & ニュートリノ反応頻度 & ニュートリノ反応数\\
& [days] & [events/day] & [events]\\
\hline
FV水あり & 56 & 199 (1490) & 11,065 (82,985)\\
FV水なし & 44 & 158 (1186) &\ 7,009 (52,570)\\
\hline
FV内 & & 41 (304) &\\
\hline \hline
\multicolumn{4}{r}{測定日数100日で計算、( )内は750kWの時}\\
\end{tabular}
\end{center}
\label{EventEstimationYear}
\end{table}%

