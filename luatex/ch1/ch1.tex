%%%%%%%%%% %%%%%%%%%% %%%%%%%%%% %%%%%%%%%% %%%%%%%%%% %%%%%%%%%%
\chapter{はじめに}
\pagenumbering{arabic}
%\pagestyle{bothstyle}
%%%%%%%%%% %%%%%%%%%% %%%%%%%%%% %%%%%%%%%% %%%%%%%%%% %%%%%%%%%%

\section{ニュートリノ}

ニュートリノは弱い相互作用を通してのみ反応する中性レプトンで、
1931年にPauliによってその存在が仮定され、
1935年にFermiによって理論付けられた粒子である。
弱い相互作用しかしないため、ニュートリノを直接検出することは非常に困難だが、
1956年にReinesとCowanによる実験によってその存在が初めて確認された。
1962年にはLederman、Schwartz、Steinbergerらの測定によって
電子の反応に現れるニュートリノとミューオンの反応に現れるニュートリノの2種類存在することが確認された。

その後、これまでに、
電子ニュートリノ($\nu_{e}$)、
ミューオンニュートリノ($\nu_{\mu}$)、
タウニュートリノ($\nu_{\tau}$)
という3世代のフレーバーとその反粒子が存在することが分かった。

ニュートリノの質量に関しては、素粒子物理学の理論である「標準模型」においてゼロとして扱われてきた。
また、直接測定による上限値が与えられるにとどまっている。
しかし近年、太陽ニュートリノや原子炉ニュートリノの観測で電子ニュートリノの数が減少する、あるいは大気ニュートリノの観測や加速器ニュートリノ実験でミューオンニュートリノの数が減少する、という現象が観測された。
これは、ニュートリノが飛行中に別のフレーバーに変わるニュートリノ振動現象によると考えられている。

ニュートリノ振動現象は、ニュートリノが質量をもつ場合にのみ起きるため、その観測は、ニュートリノが有限の質量をもつことを意味する。

%%%%%%%%%%%%%%%%%%%%%%%%%%%%%%%%%%%%%%%%%%%%%%%%%%%%%%%%%%%%

\section{ニュートリノ振動}

ニュートリノ振動とは、ある種類のニュートリノが、その時間発展とともに、
他の種類のニュートリノに変化する現象で、ニュートリノ間の質量が異なる場合にのみ起こり得る。
ニュートリノ振動現象が起きるということは、ニュートリノ質量の存在を示すとともに、
質量の固有状態がレプトン世代間で混合していることを示している。

ニュートリノ振動は、フレーバーの固有状態
(
$\nu_{e}$,
$\nu_{\mu}$,
$\nu_{\tau}$
)
と質量固有状態
(
$\nu_{1}$,
$\nu_{2}$,
$\nu_{3}$
)
が一致せず、さらに3つの質量固有状態が1つに縮退してない場合に起こる。
この場合、混合状態は3つの混合角
(
$\theta_{12}$,
$\theta_{23}$,
$\theta_{13}$
)
と1つのCP複素位相$\delta$を用いて以下の用に記述することができる。


\begin{equation}
    \begin{pmatrix}
        \nu_{e}\\
        \nu_{\mu}\\
        \nu_{\tau}
    \end{pmatrix}
    =
    \begin{pmatrix}
        U_{e1} & U_{e2} & U_{e3}\\
        U_{\mu1} & U_{\mu2} & U_{\mu3}\\
        U_{\tau1} & U_{\tau2} & U_{\tau3}
    \end{pmatrix}
    \begin{pmatrix}
        \nu_{1}\\
        \nu_{2}\\
        \nu_{3}
    \end{pmatrix}
\end{equation}


この$3 \times 3$行列は世代間の混合を表すユニタリ行列で、
MNS(Maki-Nakagawa-Sakata)行列と呼ばれ\cite{mns}、通常以下のように書かれる。

\begin{equation}
    U_{\alpha i} =
    \begin{pmatrix}
        1 & 0 & 0 \\
        0 & C_{23} & S_{23} \\
        0 & -S_{23} & C_{23} \\
    \end{pmatrix}
    \begin{pmatrix}
        C_{13} & 0 & S_{13}e^{-i\delta} \\
        0 & 1 & 0 \\
        -S_{13}e^{i\delta} & 0 & C_{13} \\
    \end{pmatrix}
    \begin{pmatrix}
        C_{12} & S_{12} & 0 \\
        -S_{12} & C_{12} & 0 \\
        0 & 0 & 1 \\
    \end{pmatrix}
\end{equation}

ここで、
$\alpha = (e, \mu, \tau)$,
$i = (1, 2, 3)$,
$C_{ij} = \cos \theta_{ij}$,
$S_{ij} = \sin \theta_{ij}$
である。

ここでは、もっとも簡単な場合として2つのフレーバー
(
$\nu_{\alpha}$,
$\nu_{\beta}$
)
間の振動を考えることにする
(
$\theta_{12} = \theta$,
$\theta_{23} = \theta_{13} = 0$
とする)。
この2つのフレーバー固有状態は質量固有状態
(
$\nu_{1}$,
$\nu_{2}$
)
を用いて

\begin{equation}
    \begin{pmatrix}
        \nu_{\alpha}\\
        \nu_{\beta}
    \end{pmatrix}
    =
    \begin{pmatrix}
        \cos\theta & \sin\theta\\
        -\sin\theta & \cos\theta
    \end{pmatrix}
    \begin{pmatrix}
        \nu_{1}\\
        \nu_{2}
    \end{pmatrix}
\end{equation}

と表すことができ、質量固有状態の時間発展は、


\begin{equation}
    \begin{pmatrix}
        \nu_{1}(t)\\
        \nu_{2}(t)
    \end{pmatrix}
    =
    \begin{pmatrix}
        e^{-i(E_{1}t-p_{1}x)} & 0\\
        0 & e^{-i(E_{2}t-p_{2}x)}
    \end{pmatrix}
    \begin{pmatrix}
        \nu_{1}(t=0)\\
        \nu_{2}(t=0)
    \end{pmatrix}
\end{equation}

と表すことができる。
ここで
$E_{i}$,
$p_{i}$
はそれぞれ
$\nu_{i}$のエネルギー、運動量を表す。

これより、フレーバー固有状態の時間発展は、

\begin{equation}
    \begin{pmatrix}
        \nu_{\alpha}(t)\\
        \nu_{\beta}(t)
    \end{pmatrix}
    = U
    \begin{pmatrix}
        e^{-i(E_{1}t-p_{1}x)} & 0\\
        0 & e^{-i(E_{2}t-p_{2}x)}
    \end{pmatrix}
    U^{-1}
    \begin{pmatrix}
        \nu_{\alpha}(t=0)\\
        \nu_{\beta}(t=0) \label{equA}
    \end{pmatrix}
\end{equation}

となる。

時刻 $t=0$、
位置 $x=0$ で生成されたニュートリノが
距離 $L$ だけ飛行した場合を考える。
ニュートリノの質量はエネルギーより
十分小さいとしてよく
($m_{i} \ll E_{i}$)

\begin{align}
    p_{i} & = \sqrt{E_{i}^{2}-m_{i}^{2}} \sim E_{i} + \frac{m_{i}^{2}}{2E_{i}}\\
    t & \sim L
\end{align}

と近似でき、\equref{equA}は、

\begin{equation}
    \begin{pmatrix}
        \nu_{\alpha}(t)\\
        \nu_{\beta}(t)
    \end{pmatrix}
    = U
    \begin{pmatrix}
        e^{-i\frac{m_{1}^{2}L}{2E_{1}}} & 0\\
        0 & e^{-i\frac{m_{1}^{2}L}{2E_{2}}}
    \end{pmatrix}
    U^{-1}
    \begin{pmatrix}
        \nu_{\alpha}(t=0)\\
        \nu_{\beta}(t=0)
    \end{pmatrix}
\end{equation}

となる。

以下ではある決まったエネルギーのニュートリノを考え、
$E_{i} = E_{\nu}$ とする。
位置 $x=0$ において $\nu_{\alpha}$ だったニュートリノが、
距離 $L$ 飛行した後に $\nu_{\beta}$ になる
確率 $P(\nu_{\alpha} \rightarrow \nu_{\beta})$、
および、 $\nu_{\alpha}$ のままである
確率 $P(\nu_{\alpha} \rightarrow \nu_{\alpha})$ はそれぞれ、

\begin{align}
    P(\nu_{\alpha} \rightarrow \nu_{\beta}) & = |\bra{\nu_{\beta}}\nu_{\alpha}\rangle|^{2} \nonumber\\
    & = \sin^{2}2\theta\sin^{2}\left(1.27\times \Delta m^{2}\ \mathrm{[eV^{2}]}\times \frac{L\ \mathrm{[km]}}{E_{\nu}\ \mathrm{[GeV]}}\right) \label{nuchange}\\
    P(\nu_{\alpha} \rightarrow \nu_{\alpha}) & = 1 - P(\nu_{\alpha}\rightarrow \nu_{\beta}) \nonumber \\
    & = 1 - \sin^{2}2\theta\sin^{2}\left(1.27 \times \Delta m^{2}\ \mathrm{[eV^{2}]} \times \frac{L\ \mathrm{[km]}}{E_{\nu}\ \mathrm{[GeV]}}\right) \label{nuunchange}
\end{align}

となる。

ここで、
$\Delta m^{2} \equiv |m_{1}^{2} - m_{2}^{2}|$ は質量二乗差のことである。

これらの式から、
確率 $P$ は
質量二乗差 $\Delta m^{2}$ と
混合角 $\theta$ をパラメータとして、
飛行距離 $L$ および
ニュートリノエネルギー $E_{\nu}$ の関数としてフレーバー間で振動することがわかる。
また、この振動が起こるのは、フレーバー固有状態が
質量固有状態と異なっており($\theta \neq 0$)、
かつ質量固有状態が縮退していない($\Delta m^{2} \neq 0$)場合であることが分かる。

このニュートリノ振動の存在を実証出来れば、
少なくとも2種類のニュートリノの間に質量差が生じることになり、
したがって、少なくとも1種類のニュートリノが質量を持つことの証明となる。


%\subsubsection{$\nu_{\mu} \rightarrow \nu_{x} \ 振動$}

%\subsubsection{$\nu_{\mu} \rightarrow \nu_{e} \ 振動$}


%%%%%%%%%%%%%%%%%%%%%%%%%%%%%%%%%%%%%%%%%%%%%%%%%%%%%%%%%%%%

\section{ニュートリノ振動実験の現状}

\if0
太陽ニュートリノ観測と原子炉ニュートリノ実験によって$\Delta m_{12}^{2}$と$\theta_{12}$が、また大気ニュートリノ観測と加速器ニュートリノ実験によって$\Delta m_{23}^{2}$と$\theta_{23}$が測定されてきた。しかし残る1つの混合角$\theta_{13}$	については、上限値$\sin^{2}\theta_{13}<0.1$が与えられているのみなので、その精密な測定結果が強く待ち望まれている。

以下に説明するT2K実験は世界最高感度のニュートリノ振動測定により、唯一未発見である$\theta_{13}$による振動モードの世界初観測を実現しようとしている。またCP位相$\delta$は$\theta_{13}$が0でないときに初めて測定可能であるため、$\theta_{13}$の測定は将来的なCP位相$\delta$の測定のためにも重要な要素となる。

現在までにSKやSNO実験による太陽ニュートリノ観測とKamLANDによる原子炉ニュートリノ実験[2]から、$\Delta m_{12}^{2}=7.9 \times 10^{-5} eV^{2}$が、また、SKによる太陽ニュートリノ観測とK2K実験による加速器ニュートリノ実験\comment{MINOSが新しい結果}から、$1.6 \times 10^{-3} < \Delta m_{23}^{2} < 3.0 \times 10^{-3} eV^{2}, \sin^{2}2\theta_{23} > 0.9$であることが分かっている。しかし、残る混合角$\theta_{13}$については原子炉を用いたCHOOZ実験による上限値$\sin^{2}2\theta_{13} < 0.1$しか分かっていない[4]。 2009年度に開始した本研究T2K実験は、off-axis ビーム法の世界初導入による世界最高感度の測定により$\sin^{2}2\theta_{13} \simeq 0.006$まで探索することを目標にしている。
\fi

ニュートリノ振動観測実験はニュートリノの生成過程によって、
大気ニュートリノ観測、太陽ニュートリノ観測、
原子炉ニュートリノ観測、および加速器ニュートリノ実験の4つに大別される。
現在までに行われたこれらの観測・実験によって分かっていることをまとめる。


\subsection{$\Delta m_{23}^{2}, \theta_{23}$ (大気ニュートリノ領域)}

$\nu_{\mu} \rightarrow \nu_{\tau}$ の振動モードに関しては、
スーパーカミオカンデによる大気ニュートリノ観測 \cite{sk_solar} で発見され、
加速器ニュートリノを用いたK2K実験 \cite{k2k} により確立された。
MINOSの最新の結果\cite{minos}によると、
振動パラメータは
$2.31 \times 10^{-3} < \Delta m_{23}^{2} < 3.43 \times 10^{-3} \mathrm{eV^{2}}$、
$\sin^{2} 2 \theta > 0.78\ (90 \%\ \mathrm{C.L.})$ である。

\subsection{$\Delta m_{12}^{2}, \theta_{12}$ (太陽ニュートリノ領域)}

$\nu_{e} \rightarrow \nu_{x}$($\nu_{e} \rightarrow \nu_{\mu}$
および$\nu_{e} \rightarrow \nu_{\tau}$)の振動モードに関しては、
スーパーカミオカンデ \cite{sk-solar}や SNO実験\cite{sno}による太陽ニュートリノ観測や
KamLAND実験\cite{kamland}による原子炉ニュートリノ観測により確立された。
振動パラメータは
$\Delta m_{12}^{2} \sim 8 \times 10^{-5}\ \mathrm{eV^{2}}$、
$\tan^{2}\theta_{12} \sim 0.5$ である。

\subsection{$\Delta m_{13}^{2}, \theta_{13}$ (原子炉ニュートリノ領域)}

$\theta_{13}$ を介した
$\nu_{\mu} \rightarrow \nu_{e}$ の振動モードは未だ見つかっておらず、
振動パラメータもCHOOZ実験による原子炉ニュートリノの観測\cite{chooz}により
$\Delta m_{23}^{2} \sim 2.4 \times 10^{-3}\ \mathrm{eV^{2}}$ のとき
$\sin^{2} 2 \theta_{13} < 0.15$ という上限値しか分かっていない。

また、CP非対称性の複素位相 $\delta$ は
MNS行列の $\sin \theta_{13}$ の項についてくるため、
$\theta_{13}$ がゼロでない時にのみ測定可能となる。
