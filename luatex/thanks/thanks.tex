%%%%%%%%%%%%%%%%%%%%%%%%%%%%%%%%%%%%%%%%%%%%%%%%%%%%
%%%%%%%%%%%%%%%%%%%%%%%%%%%%%%%%%%%%%%%%%%%%%%%%%%%%
\chapter*{謝辞}

修士課程の2年間は、とても短く感じられましたが、多くの方々に支えられて、充実した日々を送ることができました。
ここに感謝の意を表したいと思います。本当にありがとうございます。

中家剛教授、市川温子准教授、そして小林隆教授には、T2K実験という世界最先端の実験場で本研究の機会を与えてくださったことに感謝いたします。

とくに市川温子准教授には、のんびりな私を要所要所で引き締めていただき、根気よく丁寧にここまでご指導いただいたことを深く感謝しております。

坂下健助教には、本研究を進める上でとてもたくさんのご助言をいただきました。
また、研究者としての姿勢を間近で学ばせていただきました。
村上明さんには、検出器シミュレーションをはじめ、本研究の至る所でたくさんお世話になりました。
山内隆寛くんに、年末の光電子増倍管の測定を手伝っていただいておかげで、測定をほぼ終わらせることができました。本当にありがとう。

検出器の作製に関して、スズノ技研株式会社の皆様に本当に感謝しております。
できあがった検出器をはじめて見たときの嬉しさはいまでも忘れません。
本当にありがとうございます。

検出器のインストールや地下での作業には第一鉄工株式会社の皆様に大変お世話になりました。
いろいろと急なお願いが多いにもかかわらず、引き受けてくださり本当にありがとうございます。

メカサポートの田井野ご夫妻と石井さんには、光電子増倍管の準備およびコネクタ付けの際には大変お世話になりました。
本当にありがとうございます。

%この研究をみんなと一緒にできて本当に良かったと思います。

J-PARCでの生活では、大谷将士さん、木河達也くん、鈴木研人くん、矢野孝臣さん、
とたまに来る松村知恵さんのおかげで、楽しく過ごせました。
大谷さんは、何かと研究の進捗を気にかけて下さり、アドバイスなどもたくさんしていただきました。
ありがとうございます。\newline

最後に、これまで研究生活を支えてくれた家族と最愛の人に心から感謝します。本当にありがとう。\newline

\begin{flushright}
2011年1月吉日\\
髙橋将太
\end{flushright}