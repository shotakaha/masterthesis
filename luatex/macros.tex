% ===== particles =====
\newcommand{\pizero}{$\pi^{0}$}

% ===== neutrino interaction =====
\newcommand{\ccqe}{$\nu + p \rightarrow \mu + n$}
\newcommand{\ncqe}{$\nu + N \rightarrow \nu + N$}
\newcommand{\ccp}{$\nu + p \rightarrow \mu + n + \pi^{0}$}
%\newcommand{\nc1p}
%\newcommand{\dis}

% = math sequences =
% \bra and \ket must be used in math mode
%\newcommand{\bra}[1]{\langle #1 |}
%\newcommand{\ket}[1]{ | #1 \rangle}

% = units =
\newcommand{\cmcm}{cm$^{2}$}

% = Tips =
\newcommand{\figref}[1]{図\ref{#1}}
\newcommand{\tabref}[1]{表\ref{#1}}
\newcommand{\equref}[1]{式(\ref{#1})}
\newcommand{\secref}[1]{第\ref{#1}章}
\renewcommand{\bibname}{参考文献}

% = Mizuche Original=
\newcommand{\fv}{\mathrm{FV}}
\newcommand{\ov}{\mathrm{OV}}
\newcommand{\ww}{\mathrm{(w/ FVwater)}}
\newcommand{\wow}{\mathrm{(w/o FVwater)}}
\newcommand{\photon}{\mathrm{photon}}
\newcommand{\pe}{\mathrm{p.e.}}
\newcommand{\nd}{\mathrm{ND}}
\newcommand{\sk}{\mathrm{SK}}
\newcommand{\miz}{\mathrm{Miz}}

% = color =
\newcommand{\red}[1]{\textcolor{red}{\textbf{#1}}}
\newcommand{\comment}[1]{\red{#1}\footnote{\red{#1}}}
\newcommand{\memo}[1]{\footnote{\red{#1}}}

% = subfigure
\renewcommand*{\thesubfigure}{(\arabic{subfigure})}

% = subsubsubsection =
%\setcounter{secnumdepth}{6}
%\makeatletter
%\newcommand{\subsubsubsection}{\@startsection{paragraph}{4}{\z@}%
%  {1.5\Cvs \@plus.5\Cdp \@minus.2\Cdp}%
%  {.5\Cvs \@plus.3\Cdp}%
%  {\reset@font\normalsize\sffamily}
%}