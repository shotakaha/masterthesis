\documentclass[11pt]{ltjsreport}
\usepackage{luatexja}
%\documentclass[11pt, draft]{jreport}

\usepackage{geometry}	% See geometry.pdf to learn the layout options. There are lots.
\geometry{a4paper}    % ... or a4paper or a5paper or ...

%余白の設定
\setlength{\textwidth}{40\zw}
%\setlength{\textheight}{44\baselineskip}
\setlength{\textheight}{40\baselineskip}
\addtolength{\textheight}{\topskip}
%\setlength{\hoffset}{0.46cm}
%\addtolength{\textheight}{2.4cm}

\usepackage[dvipdfmx]{graphicx, color}
\usepackage{wrapfig}
%\usepackage{mediabb}
\usepackage{subfigure}
\DeclareGraphicsRule{.tif}{png}{.png}{`convert #1 `dirname #1`/`basename #1 .tif`.png}
\usepackage{wallpaper}
%\usepackage{utf}
\usepackage{ulem}
%\usepackage{umoline}
%\usepackage{epstopdf}
\usepackage{float}
\usepackage{lineno}
%\linenumbers
\usepackage{amsmath, amssymb}
\usepackage{type1cm}

\西暦

\bibliographystyle{jplain}

\usepackage[%dvipdfm,%
bookmarks=false,%
bookmarksnumbered=true,%
bookmarkstype=toc,%
colorlinks=true,%
linkcolor=black,%
citecolor=black,%
pdftitle={Mizucheの開発},
pdfauthor={Shota TAKAHASHI},
pdfkeywords={T2K, Mizuche, neutrino, Kyoto University}]{hyperref}

% bookmark(しおり)の文字化け対策
\ifnum 42146=\euc"A4A2
\AtBeginDvi{\special{pdf:tounicode EUC-UCS2}}\else
\AtBeginDvi{\special{pdf:tounicode 90ms-RKSJ-UCS2}}\fi

%%%%%%%%%%%%%%%%%%%%%%%%%%%%%%%%%%%%%%%%%%%%%%%%%%%%%%%%%%%%%%%%%%%%%%%%%%%%%%%
% ===== particles =====
\newcommand{\pizero}{$\pi^{0}$}

% ===== neutrino interaction =====
\newcommand{\ccqe}{$\nu + p \rightarrow \mu + n$}
\newcommand{\ncqe}{$\nu + N \rightarrow \nu + N$}
\newcommand{\ccp}{$\nu + p \rightarrow \mu + n + \pi^{0}$}
%\newcommand{\nc1p}
%\newcommand{\dis}

% = math sequences =
% \bra and \ket must be used in math mode
\newcommand{\bra}[1]{\langle #1 |}
\newcommand{\ket}[1]{ | #1 \rangle}

% = units =
\newcommand{\cmcm}{cm$^{2}$}

% = Tips =
\newcommand{\figref}[1]{図\ref{#1}}
\newcommand{\tabref}[1]{表\ref{#1}}
\newcommand{\equref}[1]{式(\ref{#1})}
\newcommand{\secref}[1]{第\ref{#1}章}
\renewcommand{\bibname}{参考文献}

% = Mizuche Original=
\newcommand{\fv}{\mathrm{FV}}
\newcommand{\ov}{\mathrm{OV}}
\newcommand{\ww}{\mathrm{(w/ FVwater)}}
\newcommand{\wow}{\mathrm{(w/o FVwater)}}
\newcommand{\photon}{\mathrm{photon}}
\newcommand{\pe}{\mathrm{p.e.}}
\newcommand{\nd}{\mathrm{ND}}
\newcommand{\sk}{\mathrm{SK}}
\newcommand{\miz}{\mathrm{Miz}}

% = color =
\newcommand{\red}[1]{\textcolor{red}{\textbf{#1}}}
\newcommand{\comment}[1]{\red{#1}\footnote{\red{#1}}}
\newcommand{\memo}[1]{\footnote{\red{#1}}}

% = subfigure
\renewcommand*{\thesubfigure}{(\arabic{subfigure})}

% = subsubsubsection =
%\setcounter{secnumdepth}{6}
%\makeatletter
%\newcommand{\subsubsubsection}{\@startsection{paragraph}{4}{\z@}%
%  {1.5\Cvs \@plus.5\Cdp \@minus.2\Cdp}%
%  {.5\Cvs \@plus.3\Cdp}%
%  {\reset@font\normalsize\sffamily}
%}

%%%%%%%%%%%%%%%%%%%%%%%%%%%%%%%%%%%%%%%%%%%%%%%%%%%%%%%%%%%%%%%%%%%%%%%%%%%%%%%
\begin{document}


%\ThisCenterWallPaper{.5}{fig/06_1.jpg}
\begin{titlepage}

\title{修士論文\\ニュートリノ反応数測定のための\\小型水チェレンコフ検出器"Mizuche"の開発}
\author{京都大学大学院理学研究科 物理学・宇宙物理学専攻\\
物理学第二教室 高エネルギー物理学研究室\\
%\UTF{9AD9}橋 将太
髙橋将太
}
\date{\today}

%\ThisULCornerWallPaper{.3}{fig/shota-pooh_christmas.pdf}
%\begin{figure}[!h]
%\centering
%\includegraphics[width=0.2\textwidth]{fig/06_1.jpg}
%\end{figure}

\end{titlepage}

\maketitle

\pagenumbering{roman}
%%%%%%%%%%%%%%%%%%%%%%%%%%%%%%%%%%%%%%%%%%%%%%%%%%%%%%%%%%%%%%%%%%%%%%%%%%%%%%%
\begin{abstract}
\end{abstract}

\tableofcontents
%\listoffigures
%\listoftables


%%%%%%%%%% %%%%%%%%%% %%%%%%%%%% %%%%%%%%%% %%%%%%%%%% %%%%%%%%%%
\chapter{はじめに}
\pagenumbering{arabic}
%\pagestyle{bothstyle}
%%%%%%%%%% %%%%%%%%%% %%%%%%%%%% %%%%%%%%%% %%%%%%%%%% %%%%%%%%%%
\chapter{T2K長基線ニュートリノ振動実験}

%%%%%%%%%% %%%%%%%%%% %%%%%%%%%% %%%%%%%%%% %%%%%%%%%% %%%%%%%%%%
\section{実験概要・目的}
T2K(Tokai to Kamioka)長基線ニュートリノ振動実験の概念図を\figref{T2KOverview}に示す。

T2K実験\cite{jhfnu}は茨城県東海村にあるJ-PARC大強度陽子加速器施設で
生成したミューオンニュートリノビームを岐阜県飛騨市のスーパーカミオカンデ検出器で
観測する全長295 kmの長基線ニュートリノ振動実験である。
本実験は2009年4月に稼働開始した。
T2K実験では世界最大強度のニュートリノビームと世界最大の水チェレンコフ検出器スーパーカミオカンデを用いて、

\begin{enumerate}
\item ミューオンニュートリノ消失による混合角$\theta_{23}$および$\Delta m_{23}$の精密測定
\item 電子ニュートリノ出現モードによる混合角$\theta_{13}$の世界初観測
\end{enumerate}

を世界最高感度で実現することを目標としている。

\begin{figure}[htbp]
\centering
%\includegraphics[bb=54 392 539 489, width=1\textwidth]{fig/T2KOverview.pdf}
\includegraphics[bb=131 315 483 418, width=1\textwidth]{fig/T2KOverview2.pdf}
\caption[T2K実験の概要]{T2K実験の概要。茨城県東海村のJ-PARC加速器施設で生成した人工ミューオンニュートリノを、295km離れたスーパーカミオカンデで観測し、ニュートリノ振動測定を行う。}
\label{T2KOverview}
\end{figure}

\subsection{ニュートリノ振動解析}

生成点直後と長距離飛行後のニュートリノの状態をそれぞれ前置検出器、後置検出器で測定を行う。
前置検出器での測定結果を外挿して、後置検出器の結果を予測し、
その値を後置検出器の測定結果と比較することにより、ニュートリノ振動解析を行う。
このとき、振動確率を表す\equref{nuchange}より、混合角は主にニュートリノ反応数の増減から、
質量二乗差は主にエネルギースペクトルの歪みから求められる。

T2K実験では、ニュートリノ生成点から280m下流に配置した前置検出器と、295 km離れた後置検出器にスーパーカミオカンデを使用する。
前置検出器での結果$N_{\nd}^{obs}$を外挿して、
スーパーカミオカンデでのニュートリノ反応数予測$N_{\sk}^{exp}$を求める式は次のようになる。

\begin{equation}
N_{\sk}^{exp}  =  R_{Far/Near} \times N_{\nd}^{obs}
\label{Extrapolation}
\end{equation}

ここで$R_{Far/Near}$はF/N比(Far-to-Near ratio)と呼ばれるもので、
モンテカルロシミュレーションにより求めた前置検出器、スーパーカミオカンデの、
それぞれのニュートリノ反応数$N_{\nd}^{MC}, N_{\sk}^{MC}$を用いて次式で定義される数である。

\begin{equation}
R_{Far/Near} \equiv \frac{N_{\sk}^{MC}}{N_{\nd}^{MC}} = \frac{\int \Phi_{\sk}^{MC} \times \sigma \times \epsilon_{\sk}}{\int \Phi_{\nd}^{MC} \times \sigma \times \epsilon_{\nd}}
\label{FN}
\end{equation}

ここで、右辺の各変数は以下の通りである。
\begin{description}
\item [$\blacksquare\ \Phi_{\sk, \nd}^{MC}$] $\cdots$ MCによるスーパーカミオカンデ、前置検出器でのエネルギースペクトル
\item [$\blacksquare\ \sigma$] $\cdots$ ニュートリノ反応断面積
\item [$\blacksquare\ \epsilon_{\sk, \nd}$] $\cdots$ スーパーカミオカンデ、前置検出器の検出効率
\end{description}

\equref{FN}より前置検出器のエネルギースペクトル、反応標的、検出効率などをスーパーカミオカンデのそれに近づけることで、それらに付いてくる不定性がお互い打ち消しあい、F/N比の系統誤差を小さくすることができる。その結果、スーパーカミオカンデでのニュートリノ反応数予測の精度を向上させることができる。



%%%%%%%%%% %%%%%%%%%% %%%%%%%%%% %%%%%%%%%% %%%%%%%%%% %%%%%%%%%%
\section{J-PARC加速器およびニュートリノビームライン}

\subsection{J-PARC加速器}

J-PARC加速器の構成を\figref{JPARC}に示す。
全長330 m の線形加速器リニアック(LINAC)で加速された陽子は、
全周350 m の 3 GeV陽子シンクロトロン(RCS)、
全周1600 m の陽子シンクロトロン(MR)の順に加速され、
最終的にビームエネルギー30G eV、ビーム強度750 kW (デザイン値)にまで到達する。
その後、超伝導磁石を用いた速い取り出し(FX)によって2 $\sim$ 4秒の間隔でニュートリノビームラインへと蹴り出される。
1スピルあたり8バンチ、1バンチの幅58 nsec、バンチ間隔581 nsecのビーム構造をしている。
これらのJ-PARC加速器の陽子ビームパラメータを\tabref{JPARCBeamlineSpec}にまとめた。
なお、ビームエネルギー30 GeV、ビーム強度750 kWを達成するために、今後これらのパラメータを変更する可能性もある。

\begin{figure}[htbp]
\centering
\includegraphics[bb=0 0 432 274, width=0.7\textwidth]{fig/T2KJPARCBL.jpg}
\caption[J-PARC加速器の構成]{J-PARC加速器の構成。LINAC、RCS、MRで徐々に加速された陽子は最終的に30 GeVのエネルギーに達する。(図はJ-PARC公式HPより)}
\label{JPARC}
\end{figure}

\begin{table}[htbp]
\caption[J-PARC加速器ビームパラメータのデザイン値]{J-PARC加速器ビームパラメータのデザイン値}
\begin{center}
\begin{tabular}{ccl}
\hline \hline
ビームエネルギー & 30 & [GeV]\\
ビーム強度 & 750 &[kW] (現在は約100 kW)\\
1スピル当たりの陽子数 & $3.3 \times 10^{14}$ & [pps] (現在は$7\times10^{7}$ pps)\\
スピル周期 & 2.11 & [sec] (現在は3.2 sec)\\
スピル構造 & 8 &[bunches/spill]\\
バンチ間隔 & 581 & [nsec]\\
バンチ幅 & 58 & [nsec] (現在は半値幅10 nsec)\\
\hline \hline
\multicolumn{3}{r}{pps = protons/spill}\\
\end{tabular}
\end{center}
\label{JPARCBeamlineSpec}
\end{table}%

\newpage
\subsection{ニュートリノビームライン}
ニュートリノビームラインの構成を\figref{T2KNeutrinoBeamline}に示す。MRで30GeVまで加速された陽子は、超電導磁石によって曲げられ、ニュートリノビームラインに導かれる。その後、陽子ビームは炭素標的に衝突し$\pi$中間子、K中間子を生成する。これらの荷電粒子を電磁ホーンによって収束させてから、崩壊トンネルに入射させる。崩壊トンネル内では、粒子の崩壊によってニュートリノやその他の粒子が生成する。ニュートリノ以外の粒子はビームダンプによって堰き止められるため、ニュートリノのみが、前置検出器群よびスーパーカミオカンデに向かって飛んでいくことができる。

\begin{figure}[htbp]
\centering
\includegraphics[bb=0 0 970 208, width=1\textwidth]{fig/T2KNBL2.pdf}
\caption[ニュートリノビームラインの構成]{ニュートリノビームラインの構成。MRで30 GeVまで加速された陽子は、超電導磁石によって曲げられたのち、炭素標的に衝突し荷電粒子を生成する。荷電粒子は電磁ホーンによって収束させられ、崩壊トンネル中でニュートリノへと崩壊し、前置検出器群・スーパーカミオカンデへと飛んでいく。}
\label{T2KNeutrinoBeamline}
\end{figure}

%\paragraph{炭素標的}
%\sout{T2Kニュートリノビームラインでは炭素をニュートリノ生成標的として用いている。陽子ビームと炭素の散乱で、大量の$\pi$中間子やK中間子が生成する。この$\pi$中間子やK中間子の崩壊反応を利用してニュートリノを生成する}

%\paragraph{電磁ホーン}
%\sout{大強度のニュートリノビームを作るためには、電磁ホーンという装置を使ってニュートリノビームを収束させる必要がある。T2Kビームラインには合計3つの電磁ホーンが建設されており、そえぞれに最大320kAの大電流を流すことができる。(ただし、現段階は250kAで運転)
%
%第1ホーン内部には炭素標的があり、陽子ビームが衝突して二次粒子(主に荷電パイオンと荷電ケーオン)を生成すると同時に、それらを収束させる。第2、第3ホーンはさらに生成粒子を収束させる。}

%\paragraph{崩壊空洞}
%\sout{Decay Volume。ニュートリノ生成点(炭素標的のある位置)から下流約100mには、内部を真空に引かれた空洞を用意してある。この空洞内で、$\pi$中間子は、ミューオンとミューオンニュートリノに2体崩壊する。このときのニュートリノを集めてビームとする。}


\if0  %%%%%%%%%% %%%%%%%%%%
\subsection{\sout{MUMON}}

\begin{figure}[htb]
\begin{center}
\includegraphics[bb=0 0 319 287, scale=0.5]{fig/temp.pdf}
\caption[MUMON]{MUMON}
\label{MUMON}
\end{center}
\end{figure}

\fi %%%%%%%%%% %%%%%%%%%%

%%%%%%%%%% %%%%%%%%%% %%%%%%%%%% %%%%%%%%%% %%%%%%%%%% %%%%%%%%%%
\section{T2K前置検出器群}
\subsection{INGRID}
INGRIDはニュートリノビーム軸上に置かれた検出器である。合計14個のモジュールからなる大質量の検出器である。ニュートリノビームの軸中心がどの方向を向いているのかを毎日確認することができる。


\subsection{オフアクシス検出器}
オフアクシス検出器はニュートリノ生成点から下流280 mに設置されており、ビーム軸からずれたスーパーカミオカンデ方向を向いている。
その断面図を\figref{TOAD}に示す。オフアクシス検出器はP0D、TPC、FGD、ECAL、SMRDの5つの検出器で構成される複合型検出器である。飛跡検出をメインに、運動量再構成、エネルギー再構成、粒子識別を行い、振動前のニュートリノのフラックスおよびエネルギーの測定を行う。測定結果を基にスーパーカミオカンデでのフラックスおよびエネルギースペクトルを予測する。

\begin{figure}[htbp]
\centering
%\includegraphics[bb=189 528 418 758, width=0.8\textwidth]{fig/T2KOAD.pdf}
\includegraphics[bb=0 0 873 773, width=0.7\textwidth]{fig/T2KND280.png}
%\includegraphics[bb=0 0 1213 1073, width=0.8\textwidth]{fig/T2KND280.pdf}
%\includegraphics[bb=14 14 4864 4305, clip, width=0.8\textwidth]{fig/T2KND280.eps}
\caption[T2Kオフアクシス検出器]{T2Kオフアクシス検出器}
\label{TOAD}
\end{figure}

\if0%%%%%%%%%%%%%%%%%%%%%%%%%%%%%%%%%%%%%
\paragraph{P0D(Pi-zero Detector)}
\sout{P0Dはプラスチックシンチレータと水標的で構成される検出器である。電磁石内の上流に設置されている。中性カレント反応によって生じた$\pi^{0}$粒子を測定することができる。
水と鉛が入っている?
%
$\pi^{0} \rightarrow 2 \gamma$の崩壊によって生じた$\gamma$を検出する。}

\paragraph{FGD(Fine Grained Detector)}
\sout{エフジーディー。Fine Grained Detectorの略。}

\paragraph{TPC(Time Projection Chamber)}
\sout{ティーピーシー。Time Projection Chamberの略。}

\paragraph{ECAL(Electromagnetic CALorimeter)}
\sout{ECALはプラスチックシンチレーターと鉛フィルムから構成される検出器で、P0D、FGD、TPCの周りを囲むように設置されている。
$\pi^{0}$中間子の崩壊によって生じる$\gamma$を検出する。また電子ニュートリノによる荷電カレント反応によって生じた電子も検出できる。}

\paragraph{SMRD(Side Muon Range Detector)}
\sout{電磁石の鉄ヨークの隙間にプラスティックシンチレータが挟んである。ニュートリノ反応によるミューオンで、横方向にすり抜けてしまったものを検出する。}

\paragraph{電磁石}
\sout{T2K電磁石にはCERNのUA1実験のものを使用。約2900Aの電流を流し、0.2Tの磁場をかけ、ニュートリノ反応によって生じた荷電粒子の運動量を測定する。電磁石内部の大きさは3.5m$\times$3.6m$\times$7.0m}

\comment{オフアクシス検出器について簡単にまとめる。SKとは違うんだよ、ということを書く}
\fi


\if0 %%%%%%%%%% %%%%%%%%%%
\subsection{\sout{INGRID}}
ビーム軸に沿って設置。
10m x 10mの大きさの十字型。
ビーム中心がずれていないか測定。

\begin{figure}[htb]
\centering
\includegraphics[bb=0 0 319 287, scale=0.5]{fig/temp.pdf}
\caption[INGRID]{INGRID}
\label{INGRID}
\end{figure}

\fi %%%%%%%%%% %%%%%%%%%%


\newpage
%%%%%%%%%% %%%%%%%%%% %%%%%%%%%% %%%%%%%%%% %%%%%%%%%% %%%%%%%%%%
\section{後置検出器:スーパーカミオカンデ}

T2K実験ではスーパーカミオカンデを、ニュートリノ発生点から295kmの地点に置かれた後置検出器として使用する。

スーパーカミオカンデは、岐阜県飛騨市神岡町の神岡鉱山茂住坑内に、東京大学宇宙線研究所付属の観測装置として建設された、水チェレンコフ光検出器である。宇宙線起源のミューオンによるバックグラウンドを減らすため、池の山山頂の地下1000m(2700m w.q.e)に建設された。実際に検出器付近での宇宙線ミューオンの強度は、地表での強度の約$10^{-5}$となっており、スーパーカミオカンデにおける宇宙線ミューオン事象の頻度は4Hzにまで抑えられている。

スーパーカミオカンデ検出器の全体図を\figref{SuperKamiokande}に示す。スーパーカミオカンデ検出器の本体となるタンクは、直径39.3m、高さ41.4mの円筒形をしており、その中は総質量5万トンの超純水で満たされている。タンクの内部は光学的に内タンク(直径33.8m、高さ36.2m、有効体積22.5トン)と外タンクに分けられており、内タンクには直径20インチの光電子増倍管約11200本が内向きに、外タンクには直径8インチの光電子増倍管約1900本が外向きに、それぞれ取り付けられている。

内タンクは粒子検出の主となる部分であり、タンクの中もしくは外で起こった反応により生じた荷電粒子が、水中を通過する際に放出するチェレンコフ光を、内タンク壁面に並べられた光電子増倍管で検出し、その光量・到達時間・リングパターンなどから、粒子の種類・エネルギー・発生点・運動方向などを決定する。

外タンクは、岩盤からの$\gamma$線や中性子によるバックグラウンド事象の除去および外部から入射する粒子(主に宇宙線ミューオン)や外部に抜ける粒子の識別、のために利用されている。

\begin{figure}[!htb]
\centering
\includegraphics[bb=86 439 501 740, width=1\textwidth]{fig/T2KSK.pdf}
\caption[スーパーカミオカンデ]{スーパーカミオカンデ}
\label{SuperKamiokande}
\end{figure}

現在までにスーパーカミオカンデは、そのずば抜けた性能により、太陽ニュートリノ・大気ニュートリノなどの自然から来るニュートリノを観測し、ニュートリノの質量に関する多くの情報をもたらしている

\if0
\section{T2K実験の現状}
余力があればNsk/Nndの最近の結果(INGRID, Off-Axis)に触れて、cancellationをdemonstrateする

\begin{quote}
余力があればNsk/Nndの最近の結果(INGRID, Off-Axis)に触れて、cancellationをdemonstrateする
\end{quote}

余力があればNsk/Nndの最近の結果(INGRID, Off-Axis)に触れて、cancellationをdemonstrateする
\fi
%%%%%%%%%%%%%%%%%%%%%%%%%%%%%%%%%%%%%%%%%%%%%%%%%%%%%%%%%%%%%%%%%%%%%%%%%%%%%%%%
\chapter{小型水チェレンコフ検出器 Mizuche}

%%%%%%%%%% %%%%%%%%%% %%%%%%%%%% %%%%%%%%%% %%%%%%%%%% %%%%%%%%%%
\section{Mizucheの概要}
Mizuche実験とは、J-PARC加速器によって生成された直後のニュートリノビーム中のニュートリノの個数を、
後置検出器であるスーパーカミオカンデと同じ水チェレンコフ光検出器で測定する実験である。

ニュートリノ振動測定の精密測定には、ニュートリノビームのフラックス、ニュートリノと反応標的の反応断面積、
検出器の検出効率の不定性に起因する系統誤差を低く抑えることが必要となってくる。
そこで、生成直後のニュートリノビームの性質を、前置検出器で測定し、スーパーカミオカンデに外挿することにより、
これらの系統誤差を小さく抑えることができる。
特に、スーパーカミオカンデと同じ測定原理・検出装置を持つ水チェレンコフ光検出器で測定することにより、
これらの系統誤差を削減することができる。

実際に、過去のK2K実験では1キロトンの水チェレンコフ光検出器を使用することで、
系統誤差をキャンセルした測定に大きく貢献している。
しかし、T2K実験の場合は、ニュートリノビーム強度がK2K実験よりも2桁強いため、
ニュートリノ反応レートが大きくなり、1キロトンもの大容積では、1バンチ内で多数のニュートリノが反応してしまい、
バンチ毎にイベントを区別して測定することが困難になってしまう。
そこで、本実験では容積を2.5トンと小型化し、ニュートリノ反応数を数えることに特化した検出器の開発を行うことにした。

\section{Mizucheの目的}

本実験は後置検出器であるスーパーカミオカンデと同じタイプの水チェレンコフ検出器を用いて
前置検出器部分でのニュートリノ反応数測定を行い、外挿することで、
系統誤差を低く抑えたニュートリノ反応数予測を目指す実験である。
それに向けて、本実験では次の2つを目標にしている。

\begin{enumerate}
\item 前置検出器部分で水チェレンコフ光検出器を用いたニュートリノ反応数測定\\(目標精度2\%)
\item スーパーカミオカンデでのニュートリノ反応数予測の精度向上
\end{enumerate}

第一目標として、前置検出器部分でのニュートリノ反応数測定の精度を2\%で行うことを目指すことを掲げている。
まずここまでで、T2K実験の大強度ニュートリノビームに対しても水チェレンコフ検出器が有効であることを実証する。

次の目標としては、ここまでに得られた結果を元に、本検出器をT2K前置検出器群と合わせて利用し、
スーパーカミオカンデでのニュートリノ反応数予測を行う。
ここで、\figref{MizuSKFlux}に本検出器とスーパーカミオカンデで予測されるニュートリノフラックスを示した。
このように、ほぼ同じ形のフラックス、同じニュートリノ反応標的(水)、同じ検出原理(チェレンコフ光)を用いることで、
最終的には、系統誤差を抑えた外挿を行うことにより、T2K実験の測定感度の向上に貢献したいと考えている。

\begin{figure}[htbp]
  \begin{minipage}{0.47\textwidth}
    \centering
    \includegraphics[bb=128 475 450 708, width=1\textwidth]{fig/MCNeutrinoFlux.pdf}
    \subcaption{Mizuche}
    \label{MizuFlux}
  \end{minipage}
  \hfill
  \begin{minipage}{0.47\textwidth}
    \centering
    \includegraphics[bb=255 191 822 575, width=1\textwidth]{fig/MizucheSKFlux3.pdf}
    \subcaption{スーパーカミオカンデ}
    \label{SKFlux}
  \end{minipage}
  \caption[Mizucheとスーパーカミオカンデでのニュートリノフラックス]{Mizucheとスーパーカミオカンデでのニュートリノフラックス}
  \label{MizuSKFlux}
\end{figure}



%\subsubsection{振動解析と系統誤差}
%以下に、とある振動解析の手法と、Mizucheを使用した場合に、どのような系統誤差を抑えることができるのかを示す。
%\begin{equation}
%N_{SK}^{exp}  =  R_{Far/Near} \times N_{Miz}^{obs}
%\label{Extrapolation}
%\end{equation}
%ここで、
%\begin{equation}
%R_{Far/Near} = \frac{N_{SK}^{MC}}{N_{Miz}^{MC}} = \frac{\int \Phi_{SK}^{MC} \times \sigma_{SK} \times \epsilon_{SK}}{\int \Phi_{Miz}^{MC} \times \sigma_{Miz} \times \epsilon_{Miz}}
%\label{Extrapolation2}
%\end{equation}

%\begin{itemize}
%\item $N_{SK}^{exp} \cdots $ SKでのニュートリノ反応の予測数
%\item $N_{Miz}^{obs} \cdots $ Mizucheでの実際のニュートリノ反応観測数
%\item $R_{Far/Near} \cdots$ Far - Near比
%\item $N_{SK, Miz}^{MC} \cdots$ MCのよるSK, Mizucheでのニュートリノ反応数
%\item $\Phi_{SK, Miz}^{MC} \cdots$ MCのよるSK, Mizucheでのエネルギースペクトル
%\item $\sigma_{SK, Miz} \cdots$ SK, Mizucheでのニュートリノ反応断面積
%\item $\epsilon_{SK, Miz} \cdots$ SK, Mizucheでの検出効率
%\end{itemize}

%%%%%%%%%% %%%%%%%%%% %%%%%%%%%% %%%%%%%%%% %%%%%%%%%% %%%%%%%%%%
\section{Mizucheの実験原理}

本検出器は、水中を高速で走る荷電粒子が放出するチェレンコフ光をとらえることにより粒子を検出する、
スーパーカミオカンデと同じ水チェレンコフ光検出器である。

ニュートリノが水中の水素原子核や酸素原子核と反応し荷電粒子が生成される。
その時の荷電粒子(主にミューオン)が水中を進むことによって放出されるチェレンコフ光を、
検出器の周りに配置した光電子増倍管で観測する。
%\comment{反応の絵}


本検出器は\figref{TankConcept}のような、
外タンク(直径\qty{1400}{\mm}、長さ\qty{1600}{\mm})の内側に、
一回り小さな内タンク(直径\qty{800}{\mm}、長さ\qty{1000}{\mm})を抱えた、
2層構造をしている。
内タンクの容積約\qty{0.5}{\cubic\meter}(=\qty{500}{\kg})を有効体積(fiducial volume: FV)と定義する。

FV内でのニュートリノ反応数は、
(1) FV内に水がある状態と、
(2) FV内に水がない状態の2状態で測定を行い、
その残差から求める。この測定原理の詳細については後述する。

\begin{figure}[htb]
  \centering
  %\includegraphics[bb=0 0 575 320, scale=0.5]{fig/MizucheTankConcept.pdf}
  \includegraphics[bb=0 0 1012 578, width=1\textwidth]{fig/MizucheTankConcept2.pdf}
  \caption[Mizuche検出器の概念設計図]{
    Mizuche検出器の概念設計図。
    青色:内タンク($\phi$ \qtyproduct{800 x 1000}{\mm});
    水色:外タンク($\phi$ \qtyproduct{1400 x 1600}{\mm});
    桃色:164本の3インチ光電子増倍管。
    FVの端でのニュートリノ反応によるチェレンコフ光を観測できるよう、
    外タンクと内タンクの間には\qty{300}{\mm}の領域(Outer Volume: OV)を設定した。
  }
\label{TankConcept}
\end{figure}


\subsection{チェレンコフ放射}

チェレンコフ放射とは、荷電粒子が媒質中を運動する時、
その速度が媒質中の光速度よりも速い場合に光を放射する現象である。
1934年にP. A. チェレンコフによって発見されたことからその名が付いている。

\subsubsection{チェレンコフ角とエネルギー閾値}

媒質の屈折率を$n$、荷電粒子の進行方向とチェレンコフ光の放出方向のなす角度を$\theta_{c}$とすると、
$\theta_{c}$は荷電粒子の速度$\beta c$によって決まり、以下の関係が成り立つ。%(\equref{CherenkovAngle})

\begin{equation}
\cos \theta_{c} = \frac{1}{n\beta}
\label{CherenkovAngle}
\end{equation}

チェレンコフ光は、荷電粒子の進行方向を軸とする円錐面に沿って放出される。
荷電粒子のエネルギーが十分大きく、その速度が光速に近い速度($\beta =1$)であるとき、
チェレンコフ角$\theta_{c}$は最大となる。
また、エネルギーが小さくなるにつれ、チェレンコフ角$\theta_{c}$は狭くなり、
エネルギーが低すぎるとチェレンコフ光は放出されない。
チェレンコフ光が放出される最低速度$\beta_{t}$(threshold velocity)と、
そのときのエネルギー閾値$E_{t}$(energy threshold)は次式で表すことができる(\equref{ThresholdVelocity}、\equref{EnergyThreshold})。

\begin{equation}
\beta_{t} = \frac{1}{n}
\label{ThresholdVelocity}
\end{equation}

%\begin{eqnarray}
%\frac{p_{t}}{E_{t}} & = & \frac{1}{n}\\
%\frac{\sqrt{E_{t}^{2}-m^{2}}}{E_{t}}  & = & \frac{1}{n} \\
%n^{2} (E_{t}^{2}-m^{2}) & = & E_{t}^{2} \\
%n^{2}E_{t}^{2} - n^{2}m^{2} & = & E_{t}^{2} \\
%(n^{2}-1)E_{t}^{2} & = & n^{2}m^{2} \\
%E_{t}^{2} & = & \frac{n^{2}m^{2}}{n^{2}-1}\\
%
%\end{eqnarray}

\begin{equation}
E_{t} = \frac{nm}{\sqrt{n^{2}-1}}
\label{EnergyThreshold}
\end{equation}

%\begin{eqnarray}
%p_{t} & = & \sqrt{E_{t}^{2}-m^{2}}\\
%& = & \sqrt{\frac{n^{2}m^{2}}{n^{2}-1}-m^{2}}\\
%& = & \sqrt{\frac{n^{2}m^{2}-m^{2}(n^{2}-1)}{n^{2}-1}}\\
%& = & \sqrt{\frac{m^{2}}{n^{2}-1}}\\
%& = & \frac{m}{\sqrt{n^{2}-1}} \ (= \beta_{t} E_{t})
%\end{eqnarray}

水の場合、屈折率$n\sim1.33$なので、最大チェレンコフ角$\theta_{c} \sim 42^{\circ}$、$\beta_{t} \sim 0.75$となる。
また、\tabref{ThresholdByParticle}に主な粒子別のエネルギーと運動量の閾値をまとめた。


\begin{table}[htbp]
\caption[主な粒子の水に対するチェレンコフ光放出のエネルギー閾値と運動量閾値]{主な粒子の水に対するチェレンコフ光放出のエネルギー閾値$E_{t}$と運動量閾値$p_{t}$}
\begin{center}
\begin{tabular}{c|ccc}
\hline \hline
& 静止質量 $m$ [MeV/c$^{2}$] & $E_{t}$ [MeV] & $p_{t}$ [MeV/c]\\
 \hline
e$^{\pm}$	& 0.511	& 0.775 & 0.583\\
$\mu^{\pm}$	& 105.7 & 160.3 & 120.5\\
$\pi^{\pm}$	& 139.6 & 211.7 & 159.2 \\
p$^{+}$	& 938.2	& 1423 & 1070\\
\hline \hline
\end{tabular}
\end{center}
\label{ThresholdByParticle}
\end{table}%

\subsubsection{単位長さあたりに放出されるチェレンコフ光子数}

荷電粒子の電荷が$ze$ [C]であるとき、単位飛程、単位波長あたりに放出される光子数$N_{\photon}$は次のように表すことができる(\equref{dNdXdL})。

%\begin{eqnarray}
%\frac{d^{2}N_{photon}}{dxd\lambda} & = & \frac{2 \pi \alpha z^{2}}{\lambda^{2}} \left( 1 - \frac{1}{\beta^{2}n^{2}(\lambda)}\right) \\
%& = & \frac{2 \pi \alpha z^{2}}{\lambda^{2}} \sin^{2} \theta_{c}
%\end{eqnarray}

\begin{equation}
\frac{d^{2}N_{\photon}}{dxd\lambda} =  \frac{2 \pi \alpha z^{2}}{\lambda^{2}} \sin^{2} \theta_{c}
\label{dNdXdL}
\end{equation}
ここで、$\lambda$はチェレンコフ光の波長、$\alpha \simeq 1/137$は微細構造定数である。
これを波長で積分すると、次式が得られる(\equref{dNdX})。

\begin{equation}
\frac{dN_{\photon}}{dx} =  2 \pi \alpha z^{2} \sin^{2} \theta_{c} \left( \frac{1}{\lambda_{1}}-\frac{1}{\lambda_{2}} \right) \ , (\lambda_{1} < \lambda_{2})
\label{dNdX}
\end{equation}

これまでの式から荷電粒子がエネルギーを失うに従って、
チェレンコフ角$\theta_{c}$が小さくなると共に、チェレンコフ光の強度も減少していくことが分かる。

典型的な光電子増倍管で検出可能な波長は\qty{300}{\nm} $\sim$ \qty{650}{\nm}である。
この範囲を考慮すると、運動量\qty{1}{\GeV/c}の荷電粒子が単位長さ進むときに放出する光子数は
$N_{\photon}/dx \sim 823\sin^{2}\theta_{c}\ $ [photon/cm]程度と見積もることができる。
ミューオン、電子のそれぞれに対して、運動量とチェレンコフ角の関係と、
荷電粒子が単位長さ進むあたりに放出されるチェレンコフ光子数の関係を
それぞれ\figref{MizucheCheDeg}と\figref{MizuchedPhoton}に図示した。

また、本検出器表面積の約6\%が光電子増倍管で覆われていることと、
光電子増倍管の量子効率が約20\%であると仮定して、荷電粒子が単位長さ進んだときに期待される光量(光電子数)を
見積もったものを\figref{MizuchedPE}に図示する。
これから\qty{1}{\cm}あたり$3\sim4$ photon しか放出されないことが分かる。
そのため、内タンク(FV)の外側に\qty{300}{\cm}のバッファー層(OV)を設けることで、
FVで生じた荷電粒子が、検出するのに十分な光量のチェレンコフ光を放射することを保証した。
\qty{30}{\cm}進んだ際に検出できる光量を\figref{MizuchePE}に示す。
これらの検出器設計詳細については後述する。

\begin{figure}[htbp]
  \begin{minipage}{0.47\textwidth}
    \centering
    \includegraphics[bb=0 0 500 484, width=1\textwidth]{fig/MizucheCheDeg.pdf}
    \caption[ミューオンと電子の運動量とチェレンコフ角の関係]{ミューオンと電子の運動量とチェレンコフ角の関係。黒線はミューオン、赤線は電子を表す。}
    \label{MizucheCheDeg}
  \end{minipage}
  \hfil
  \begin{minipage}{0.47\textwidth}
    \centering
    \includegraphics[bb=0 0 500 484, width=1\textwidth]{fig/MizuchedPhoton.pdf}
    \caption[ミューオンと電子の運動量と単位飛程あたりに放出されるチェレンコフ光子数の関係]{ミューオンと電子の運動量と単位飛程あたりに放出されるチェレンコフ光子数の関係。黒線はミューオン、赤線は電子を表す。}
    \label{MizuchedPhoton}
  \end{minipage}
\end{figure}

\begin{figure}[htbp]
  \begin{minipage}{0.47\textwidth}
    \centering
    \includegraphics[bb=0 0 500 484, width=1\textwidth]{fig/MizuchedPE.pdf}
    \caption[Mizucheで検出できる単位飛程あたりの光電子数]{Mizucheで検出できる単位飛程あたり光電子数。光電被覆率 6.24\%、量子効率19\%とした。黒線はミューオン、赤線は電子を表す。}
    \label{MizuchedPE}
  \end{minipage}
  \hfil
  \begin{minipage}{0.47\textwidth}
    \centering
    \includegraphics[bb=0 0 500 484, width=1\textwidth]{fig/MizuchePE.pdf}
    \caption[荷電粒子が 300 mm 進むときに期待される光電子数数]{荷電粒子が 300 mm 進むときに期待される光電子数。黒線はミューオン、赤線は電子を表す。}
    \label{MizuchePE}
  \end{minipage}
\end{figure}

\newpage

\subsection{測定原理}

前述したように、本検出器は、次の2状態で測定を行い、その残差を求める。

\begin{enumerate}
\item FV内に水がある状態(FV水あり)
\item FV内に水がない状態(FV水なし)
\end{enumerate}

この測定原理について、\figref{EventCategory}を用いて詳しく説明する。
上段の図は(1)FV内に水がある状態での測定、
下段の図は(2)FV内に水がない状態での測定を表している。
それぞれの場合でチェレンコフ光が発生する要因によって4つに場合分けして図示した。

\begin{figure}[htbp]
\centering
\includegraphics[bb=16 103 971 616, width=1\textwidth]{fig/MizucheEventCategory.pdf}
\caption[測定原理の概略]{測定原理の概略。上段:FV水ありの測定、下段:FV水なしの測定。チェレンコフ光が発生する要因によって4つに場合分けした。}
\label{EventCategory}
\end{figure}

\subsubsection{1. FV内で起こるニュートリノ反応(左端の図)}

本検出器のシグナルイベントである。
FV内でのニュートリノ反応は、FV水ありの状態でしか起こらないため、その残差はFV内での反応数、すなわちシグナルイベントとなる。


\subsubsection{2. FV外で起こるニュートリノ反応(左から2番目の図)}
FV外(OV)でのニュートリノ反応は、FV水あり、FV水なしとも起こるため、
両状態の検出効率が全く等しい場合、差をとれば反応数は相殺する。
相殺しなかった場合は、バックグラウンドとなる。
それぞれの場合で期待される検出効率については、\secref{MonteCalro}の検出器シミュレーションにて詳述する。

\subsubsection{3. 砂ミューオンによるチェレンコフ放射(左から3番目の図)}

砂ミューオンが発生するチェレンコフ光によるイベントである。
砂ミューオンとは、前置検出器ホールの壁とニュートリノが反応したことにより生じたミューオンのことである。
このイベントはFV水あり、水なしでも起こるため、OVで起こるニュートリノ反応同様、差をとれば反応数は相殺する。
相殺しなかった場合はバックグラウンドとなる。

\subsubsection{4. 検出器外からの中性粒子による反応(右端の図)}

砂ミューオンの発生同様、前置検出器ホールの壁とニュートリノが反応したことによる中性粒子(主に中性子)が、
検出器内の水と反応し、荷電粒子を生成するイベントである。
FV内でこの反応が生じた場合、差をとるとバックグラウンドとして残ることになる(OVで生じた反応は相殺する)。
中性子による反応がどの程度起きるかは検出器シミュレーションにより見積もる。

%%%%%%%%%%%%%%%%%%%%%%%%%%%%%%%%%%%%%
%%%%%%%%%%%%%%%%%%%%%%%%%%%%%%%%%%%%%
%%%%%%%%%%%%%%%%%%%%%%%%%%%%%%%%%%%%%
\section{簡単なニュートリノ反応数の見積もり}

\subsection{有効体積内でのニュートリノ反応頻度}

T2Kニュートリノビームのデザイン強度で、FV内でのニュートリノ反応数を見積もった。
ニュートリノビームのフラックスを$\Phi_{\nu}$、反応断面積を$\sigma_{\nu}$、標的粒子数を$n$とすると、
ニュートリノ反応数$N_{\nu}$は次の式で表すことができる。

\begin{equation}
N_{\nu}  = \Phi_{\nu} \times \sigma_{\nu} \times n
\end{equation}

陽子ビーム強度\qty{750}{\kW}の時のニュートリノフラックス $\Phi_{\nu}=$ \qty{1.85e6}{\per \cm\squared \per\second}、
反応断面積 $\sigma_{\nu}=$ \qty{0.63e-38}{\per \cm\squared / nucleon}、
水\qty{500}{\kg}中の核子数 $n=$ \qty{3.01e29}{nucleon}より、
ニュートリノ反応数 $N_{\nu}=$ \qty{3.5e-3}{events/\second}が期待されると見積もった。
\tabref{EventRateEstimation}に計算に用いた条件をまとめた。

\begin{table}[htbp]
\caption{ニュートリノ反応数の見積もりに使用した条件}
\begin{center}
\begin{tabular}{ccl}
\hline \hline
陽子ビームエネルギー & 30 & [GeV]\\
%陽子ビーム数 & $3.3 \times 10^{14}$ & [/\cmcm/spill]\\
陽子ビーム強度 & 0.75 &[MW] \\
%ビーム周期(スピル間隔) & 2.11 & [sec] \\
%1スピル当たりの陽子数 & $3.3 \times 10^{14}$ & [protons/spill]\\
%ニュートリノビームフラックス & $6.5 \times 10^{6}$ & [/\cmcm /spill] \\
ニュートリノビームフラックス & $1.85 \times 10^{6}$ & [/\cmcm /sec] \\
ニュートリノエネルギー & 0.7 & [GeV]\\
水反応断面積 & $0.63 \times 10^{-38}$ & [\cmcm/nucleon]\\
FV質量 & 0.5 & [ton] \\
核子数 & $3.01\times10^{29}$ & [nucleon]\\
\hline
ニュートリノ反応頻度 & $3.50 \times 10^{-3}$ & [events/sec]\\
\hline \hline
\end{tabular}
\end{center}
\label{EventRateEstimation}
\end{table}%

\subsection{測定時間による統計誤差}

限られたビームタイムの中で、FV内に水がある状態とない状態の2状態の測定を行わなければならない。
そこで、それぞれの測定時間による統計誤差が最小となるよう、最適な水あり/水なしの測定時間の比を見積もった。

ニュートリノフラックスを$\Phi_{\nu}$、反応断面積を$\sigma_{\nu}$、アボガドロ数を$N_{A}$とし、
FV水ありの時の体積を$V_{1}$・測定時間を$T_{1}$・ニュートリノ反応数を$N_{1}$、
FV水なしでの体積を$V_{2}$・測定時間を$T_{2}$・ニュートリノ反応数を$N_{2}$とすると、
それぞれの状態でのニュートリノ反応数は次のようになる。

\begin{description}
\item [FV水あり]%
\begin{equation}
%N_{1}=\sigma_{\nu}F_{\nu}n_{1}T_{1}=\sigma_{\nu}F_{\nu}N_{A}V_{1}T_{1}
N_{1} = \sigma_{\nu}\Phi_{\nu}N_{A}V_{1}T_{1}
\label{Nww}
\end{equation}
%
\item [FV水なし]
\begin{equation}
%N_{2}=\sigma_{\nu}F_{\nu}n_{2}T_{2}=\sigma_{\nu}F_{\nu}N_{A}V_{2}T_{2}
N_{2} = \sigma_{\nu}\Phi_{\nu}N_{A}V_{2}T_{2}
\label{Nwow}
\end{equation}
\end{description}

ここで、測定時間の比を$T_{1}:T_{2}=a:b$とすると、FV内でのニュートリノ反応数は次のようになる
(測定時間をFV水ありに合わせた)。

\begin{equation}
N_{\fv} = N_{1}-\frac{T_{1}}{T_{2}}N_{2} = N_{1}-\frac{a}{b}N_{2}
\label{Nfv}
\end{equation}

$N_{1}$、$N_{2}$はポアソン過程だと仮定すると、それぞれの統計誤差は次のようになる。
\begin{eqnarray}
\sigma_{N_{1}} & = & \sqrt{N_{1}} \label{sigma1}\\
\sigma_{N_{2}} & = & \sqrt{N_{2}} \label{sigma2}
\end{eqnarray}

また、誤差の伝播式より、$N_{\fv}$の統計誤差は次のようになる。
\begin{equation}
\sigma_{N_{\fv}} = \sqrt{\sigma_{N_{1}}^{2}+\left(\frac{a}{b}\right)^{2}\sigma_{N_{2}}^{2}} \label{sigmafv}
\end{equation}

全体に対する統計誤差の割合を計算すると、
\begin{eqnarray}
\frac{\sigma_{N_{\fv}}}{N_{\fv}} & = & \frac{\sqrt{\sigma_{N_{1}}^{2}+\left(\frac{a}{b}\right)^{2}\sigma_{N_{2}}^{2}}}{N_{1}-\frac{a}{b}N_{2}}\\
%
& = & \frac{\sqrt{N_{1} + \left(\frac{a}{b}\right)^{2}N_{2}}}{N_{1}-\frac{a}{b}N_{2}}\\
& = & \frac{1}{\sqrt{N_{1}}} \cdot%
\frac{1}{\sqrt{V_{1}}} \cdot%
\frac{\sqrt{V_{1}+\frac{a}{b}V_{2}}} {V_{1}-V_{2}}\\
\end{eqnarray}

ここで、$\sqrt{V_{1}+\frac{a}{b}V_{2}}$が最小となる$a$、$b$を考える。
\\
\\
相加平均・相乗平均の定理より、
\begin{eqnarray}
A + B & \ge & 2\sqrt{AB} \ \text{(等号成立は$A=B$)} \label{sksj}\\
V_{1} + \frac{a}{b}V_{2} & \ge & 2\sqrt{V_{1}\frac{a}{b}V_{2}}\\
\text{等号成立は} & & V_{1} = \frac{a}{b}V_{2} \Rightarrow a:b=V_{1}:V_{2}\\
\text{(このとき} & & V_{1}+\frac{a}{b}V_{2} = 2V_{1}\text{\ )}
\end{eqnarray}

したがって、
\begin{equation}
T_{1}:T_{2} = V_{1}:V_{2}
\end{equation}
測定時間による統計誤差を最小にするためには、測定時間を測定状態の体積比に配分すれば良いことが分かった。

\subsection{1年間で期待されるニュートリノ反応数}
1年のビームタイムを100日と仮定し、前節のとおりに測定時間を配分したときに期待されるニュートリノ反応数を見積もった。FV水ありの体積は2.5トン、FV水なしの体積は2.0トンであるので、測定時間はそれぞれ56日と44日になる。ビーム強度を100 kW(750 kW)と仮定したときのイベント数を\tabref{EventEstimationYear}にまとめた。FV内でのニュートリノ反応は1日あたり41(304)イベントが期待できる。


\begin{table}[htbp]
\caption[期待されるニュートリノ反応数]{期待されるニュートリノ反応数}
\begin{center}
\begin{tabular}{cccc}
\hline \hline
測定状態 & 測定日数 & ニュートリノ反応頻度 & ニュートリノ反応数\\
& [days] & [events/day] & [events]\\
\hline
FV水あり & 56 & 199 (1490) & 11,065 (82,985)\\
FV水なし & 44 & 158 (1186) &\ 7,009 (52,570)\\
\hline
FV内 & & 41 (304) &\\
\hline \hline
\multicolumn{4}{r}{測定日数100日で計算、( )内は750kWの時}\\
\end{tabular}
\end{center}
\label{EventEstimationYear}
\end{table}%


\include{ch4/ch4.tex}
\include{ch5/ch5.tex}







%%%%%%%%%%%%%%%%%%%%%%%%%%%%%%%%%%%%%%%%%%%%%%%%%%%%%%%%%%%%%%%%%%%%%%%%%%%%%%%%
%%%%%%%%%%%%%%%%%%%%%%%%%%%%%%%%%%%%%%%%%%%%%%%%%%%%%%%%%%%%%%%%%%%%%%%%%%%%%%%%
\chapter{光電子増倍管のキャリブレーション}
\label{PMTCalibration}
\section{目的}
本実験では直径3インチの光電子増倍管を164本使用する。\secref{MonteCalro}で説明したような光量カットによるイベント選択をうまく行うためには、ある入射光量に対して全ての光電子増倍管から一様の応答が返ってくる必要がある。

そのため、光電子増倍管間の相対的量子効率と、各光電子増倍管の電流増幅率曲線の測定を予め行うことにより、各光電子増倍管間の量子効率の違いを把握することと、印加電圧を制御することで、電流流増幅率が調整可能なようにした。



\section{測定原理}

\subsection{光電子数と電流増幅率}

光子が発生し光電子増倍管の光電面に入射し、光電子に変換される過程はポアソン分布に従う。
平均入射光電子数を$\lambda_{\pe}$、光電子増倍管の電流増幅率を$G$とすると、測定出来る平均信号$\mu$は
\begin{equation}
\mu = G\cdot e \cdot\lambda_{\pe}
\label{mu}
\end{equation}
となる。ここで$e$は素電荷である。

このとき、電流増幅率$G$は常に一定であり、信号の標準偏差$\sigma_{\mu}$は平均入射光電子数$\lambda$のポアソンゆらぎによることとなり、
\begin{equation}
\sigma_{\mu} = Ge\cdot \sigma_{\lambda_{\pe}}=Ge\cdot \sqrt{\lambda_{\pe}}
\label{sigmamu}
\end{equation}
が成り立つ。上2式より、入射光電子数$\lambda_{\pe}$、電流増幅率$G$が計算でき、
\begin{eqnarray}
\lambda_{\pe} = \left(\frac{\mu}{\sigma_{\mu}}\right)^{2}
&, & G  =  \frac{1}{e}\cdot \frac{\sigma_{\mu}^{2}}{\mu} \label{pegain}
\end{eqnarray}
となる。

今回の測定で得られるADCのヒストグラムの平均値Mean、と標準偏差RMS\footnote{本来、標準偏差とRMSは異なるものであるが、今回用いた解析ツールROOTでは、標準偏差をRMSと表記しているので、それに従うことにする}の関係は、AD変換係数$C_{AD}$とすると、次のようになり、
\begin{equation}
\mu = C_{AD} \times Mean,\  \sigma_{\mu} = C_{AD} \times RMS
\end{equation}
これらを\equref{pegain}に代入すると、
\begin{eqnarray}
\lambda_{\pe} & = & \left(\mathrm{\frac{Mean}{RMS}}\right)^{2}\\
G & = & \mathrm{\frac{RMS^{2}}{Mean}} \times \frac{C_{AD}}{e} \label{adcrms}
\end{eqnarray}
が得られる。
このようにして、入射光電子数$\lambda_{\pe}$と電流増幅率$G$を計算した。

本測定では使用したCAMAC ADC(LeCroy 2249W)のスペック値$C_{AD}=0.25$ [pC/count]を使用した。

\subsection{相対的量子効率}
光源から放出され、光電面に入射した光子数を$\lambda_{\photon}$、光電面の量子効率を$Q$とすると、光電面から放出される光電子数は$\lambda_{\pe}=Q \cdot \lambda_{\photon}$である。入射光子数の絶対値が分かっていれば、$Q$を求めることができるが、今回の測定では入射光子数の絶対値が分からないため、下記のようにして相対的量子効率を求めることにした

基準となる光電子増倍管の量子効率を$Q^{(ref)}$、測定した$i$番目の光電子増倍管の量子効率を$Q^{(i)}$とすると、同量の光子が入射した時のそれぞれの光電子増倍管で測定される光電子数は次のようになる。
\begin{eqnarray}
\lambda_{\pe}^{(ref)} & =  & Q^{(ref)}\lambda_{\photon}\\
\lambda_{\pe}^{(i)} & = & Q^{(i)}\lambda_{\photon}
\end{eqnarray}
よって、相対的量子効率$Q_{rel}^{(i)}$は
\begin{equation}
Q_{rel}^{(i)} \equiv \frac{Q^{(i)}}{Q^{(ref)}}%
 = \frac{Q^{(i)}\lambda_{\photon}}{Q^{(ref)}\lambda_{\photon}} %
 = \frac{\lambda_{\pe}^{(i)}}{\lambda_{\pe}^{(ref)}}%
\end{equation}
となる。


\subsection{電流増幅率曲線}
光電子増倍管の二次電子放出比$\delta$はダイノード間電圧$E$の関数となり次のように表すことができる。
\begin{equation}
\delta = a \cdot E^{k}
\label{SecPEratio}
\end{equation}
ここで、$a$は定数、$k$は電極の構造・材質で決まる数である(通常$k=0.7\sim0.8$程度)。

等分割デバイダの場合、印加電圧を$V$、ダイノード段数を$n$とすると、各ダイノード間の電圧$E$は次のようになり(\equref{DinodeVoltage})、
\begin{equation}
E=\frac{V}{n+1}
\label{DinodeVoltage}
\end{equation}
電流増幅率$G$は、
\begin{eqnarray}
G & = &  \delta^{n} = aE^{k} \cdot aE^{k} \cdots aE^{k}  \nonumber \\
%& = & (aE^{k})^{n} \\
& = & a^{n} \left(\frac{V}{n+1}\right)^{kn}  \nonumber \\
& = & A \cdot V^{B}
\ \ \ \left(\ A \equiv \frac{a^{n}}{(n+1)^{kn}}, B\equiv kn  \text{とした}\right) \label{fit1}
%G & = & A \cdot V^{kn}
\end{eqnarray}
となり、印加電圧$V$の冪関数で表すことができる。
また、この両辺の対数をとると、
\begin{eqnarray}
\log_{10}G & = & \log_{10}A + B \log_{10}V \label{fit2}
\end{eqnarray}
となり、両対数目盛上で直線となる。
電流増幅率曲線の解析では\equref{fit2}を使ってフィッティングすることでその係数を求めた。


\section{方法・手順}
\subsection{実験器具}
測定の際のセットアップの概略図を\figref{CalibSetUpFig}に示す。
光電子増倍管やLEDなど必要な装置は\figref{CalibPMT}のように暗箱の中に設置した。

光電子増倍管はLEDを中心に8個セットした。それぞれ予め取り付けてあるジグにマジックテープで固定できるようになっている。設置場所は、\figref{CalibSetUpFig}のCH1の位置を基準に反時計回りにCH2〜CH8と呼ぶことにする。

LEDには青色光のものを使用した。この青色光の波長は470 nm程度であり、これはチェレンコフ光や、光電面の感度波長ピークに近い波長である。

LEDは暗箱の中心に、上を向けた状態でセットした。またその高さを光電面の中心になるように台座を調整した。LED光は指向性が強いため、光をできるだけ等方的に散乱させるためのキャップを取り付けた。\figref{CalibLED}にLEDをセットした状態を示す。

\begin{figure}[htbp]
\centering
%\includegraphics[bb=0 0 905 410, width=1\textwidth]{fig/CalibSetUpFig.pdf}
\includegraphics[bb=0 0 1010 458, width=0.78\textwidth]{fig/CalibSetUpFig2.pdf}
\caption[測定セットアップ概略図]{測定セットアップ概略図。パルスジェネレータの信号をトリガにして、LEDを光らせるとともに、ADCのゲートを開き、光電子増倍管からの信号を取得する。}
\label{CalibSetUpFig}
\end{figure}

\begin{figure}[!htbp]
  \begin{minipage}{0.47\textwidth}
    \subfigure[暗箱の中にLEDと光電子増倍管をセットした。LEDを中心に、8本の光電子増倍管を設置し、それぞれの光電子増倍管はマジックテープで固定する。]{
    %\includegraphics[bb=0 0 400 300, width=1\textwidth]{fig/P1090975.JPG}
%\includegraphics[bb=0 0 400 300, width=1\textwidth]{fig/P1090976.JPG}
%\includegraphics[bb=0 0 400 300, width=1\textwidth]{fig/P1090977.JPG}
\includegraphics[bb=0 0 400 267, width=1\textwidth]{fig/IMG_0254.JPG}
   \label{CalibPMT}}
  \end{minipage}
  \hfill
  \begin{minipage}{0.47\textwidth}
    \subfigure[LEDには散乱キャップ(図手前の白い被せ物)を取り付け、光が等方的に放出されるようにした。また高さがほぼ光電面の中心に来るようにした。]{
\includegraphics[bb=0 0 400 267, width=1\textwidth]{fig/IMG_0258.JPG}
   \label{CalibLED}}
  \end{minipage}
    \caption{実際の測定セットアップ}
  \label{CalibSetUpPic}
\end{figure}


%今回のキャリブレーションではLEDを光源として利用した。LEDは日亜化学工業のものを使用した。幅20nsのパルスを200Hzで入力しLEDを点滅させる。それと同期してADCのGateを開き、光電子増倍管からの信号を測定。CAMAC ADCにはLeCroy2249Wを使用した。

\if0
\begin{table}[htdp]
\caption{キャリブレーションに使用した装置一覧}
\begin{center}
\begin{tabular}{cc}
\hline \hline
LED & NICHIA \\
LED光拡散キャップ & OPTOSUPPLY\\
CAMAC ADC LeCroy 2249W\\
\hline \hline
\end{tabular}
\end{center}
\label{default}
\end{table}%
\fi


\subsection{基本的な測定手順}

パルスジェネレータからの信号をトリガにして、LEDとゲートジェネレータに入力することによって、LEDが点灯したタイミングで光電子増倍管の信号をCAMAC ADCを使って読み取った。

パルスジェネレータから、幅20 ns、周波数200 Hz、高さ3.2 Vのパルス信号を入力しLEDを点灯させた。またパルス信号と同期して出力されているTTL信号を、レベルアダプターを通しNIM信号へ変換させた後、ゲートジェネレータへと入力した。ゲートジェネレータからは幅60 nsのゲート信号を出力し、CAMAC ADCに入力した。

シグナル、ベデスタルそれぞれ10000イベントのデータを取得した。
ペデスタルの測定はLEDへのパルスをOFFにした状態\footnote{この状態でも200HzでTTL信号が出ている}で行い、シグナルを測定する前に毎回行った。

\figref{CalibMeasurement}にオシロスコープで確認した信号の波形と、その測定の時に得られたADC分布を示す。

\begin{figure}[htbp]
%%%
  \begin{minipage}{0.47\textwidth}
    \subfigure[オシロスコープで確認した波形。上から順にCH1、CH2、CH3の光電子増倍管からの出力、一番下がLEDへのパルスと同時に開いたゲート]{
\includegraphics[bb=0 0 800 480, width=1\textwidth]{fig/CalibOscillo.pdf}
   \label{CalibOscillo}}
  \end{minipage}
\hfill%%%
  \begin{minipage}{0.47\textwidth}
    \subfigure[左図のときに得られたCH1の光電子増倍管のADC分布。黒線:実測値、赤線:ガウス関数でフィットした曲線]{\includegraphics[bb=255 36 822 575, width=1\textwidth]{fig/RUN7_CH1_HV1100_SIG.pdf}
   \label{CalibData}}
  \end{minipage}
    \caption[測定データ]{測定データ}
  \label{CalibMeasurement}
\end{figure}


\section{等方性・再現性の測定}
全ての光電子増倍管の相対的量子効率・電流増幅率曲線の測定を行う前に、前述した設定の下、LEDから放出される光量の等方性、また本セットアップの測定の再現性を確認した。

まず、光電子増倍管を適当に2本選択した。1本(PMT1)はLEDからの光量をモニターする参照用として\figref{CalibSetUpFig}のCH1に固定し、残りの1本(PMT2)を、CH2からCH8まで順番に移動させて、光量の測定を行った。このとき、PMT1、PMT2の印加電圧はともに1100 Vに設定した。

実験原理で説明した方法で求めた各位置(=CH)での光電子数を、PMT1の光電子数で規格化する。
\begin{equation}
\text{規格化した光電子数}R_{\pe} \equiv \frac{\text{PMT2で観測した光電子数}}{\text{PMT1で観測した光電子数}}
\end{equation}
PMT1の光電子数で規格化することにより、LEDから放出される光子数の測定ごとの不定性を抑えるようにした。

各CHでの$R_{\pe}$を比較することによって、光量の等方性を確認した。また、CH2からCH8までの測定を1回として複数回測定を行うことにより、本セットアップの再現性を確認した。



\subsection{等方性と再現性}
上記の測定を5回測行った結果を\figref{CalibIsoRepCheck}にまとめた。横軸をCH番号(光電子増倍管を固定した場所)、縦軸を$R_{\pe}$にしてプロットした。線の色の違いは、それを測定したセット番号を表している。

\figref{CalibIsoRepCheck}から、各CHでの5回の測定結果は統計誤差の範囲でほとんど一致していることが分かる。これより、LEDからの光量は等方的ではないが、再現する位置依存性があることが分かる。そこで、位置による光量の違いを補正する数を次に述べるように定義した。

\begin{figure}[htbp]
\begin{center}
%\includegraphics[bb=0 0 779 624, width=0.8\textwidth]{fig/CalibIsoRepCheck.pdf}
\includegraphics[bb=255 36 822 575, width=0.8\textwidth]{fig/ANA20_PENORM.pdf}
\caption[等方性と再現性の確認]{等方性と再現性の確認。横軸に光電子増倍管を置いた場所、縦軸に規格化した光電子数をプロットした。線の色は測定セットの違いを表す。5セット分の測定が同じような分布をしていることから位置依存性があることが分かる}
\label{CalibIsoRepCheck}
\end{center}
\end{figure}

\subsubsection{場所による光量補正係数}
光電子増倍管の設置場所による光量の補正係数を\tabref{PECorrectionFactor}にまとめた。
場所による光量補正係数は\figref{CalibIsoRepCheck}の5回の測定で得られた設置場所ごとの$R_{\pe}$の平均値で定義した。
また再現性は、その平均値周りの標準偏差の大きさの割合で評価した。相対的量子効率を求める際は、この補正係数を考慮して算出する。

\begin{table}[htbp]
\caption{光電子増倍管設置場所での光量補正係数}
\begin{center}
\begin{tabular}{cccccccc}
\hline \hline
設置場所 & 2 & 3 & 4 & 5 & 6 & 7 & 8\\
\hline
%補正係数 &1.21 & 1.32 & 1.46 & 1.46 & 1.44 & 1.30 & 1.24\\
%再現性(\%) & 3.5 & 1.8 & 2.3 & 1.7 & 3.1 & 3.2 & 3.5\\
補正係数 &1.22 & 1.33 & 1.46 & 1.45 & 1.46 & 1.30 & 1.25\\
再現性(\%) & 2.40 & 1.80 & 2.02 & 2.15 & 3.30 & 4.05 & 2.99\\
\hline \hline
\end{tabular}
\end{center}
\label{PECorrectionFactor}
\end{table}%


\section{相対的量子効率・電流増幅率曲線の測定}

等方性・再現性確認のときに使用した光電子増倍管(PMT1)を引き続き光量モニター用光電子増倍管として使用し、154本の光電子増倍管の測定を行った。基本的な測定手順はこれまでと同様である。

ただし、今回は一度に7本ずつ(PMT1は除く)測定を行い、印加電圧に対する電流増幅率を調べるために、印加電圧を1000 Vから1300 Vまで50 V刻みで変化させて測定を行った\footnote{PMT1は光量モニターの役割があるため常に1100 Vを印加した}。7本の光電子増倍管を、1000 Vから1300 Vまで測定することを1ランと呼ぶことにする。

\subsection{相対的量子効率}

\begin{figure}[!h]
\centering
%%%
  \begin{minipage}{0.8\textwidth}
    \subfigure[光量補正前:Mean=1.239, RMS=0.1778]{
    \includegraphics[bb=255 306 822 575, width=1\textwidth]{fig/H1_PE_NORM_HV1100.pdf}
%    \includegraphics[bb=255 36 822 575, width=1\textwidth]{fig/H1_PE_NORM_HV1100v2.pdf}
%    \includegraphics[bb=255 36 822 575, width=1\textwidth]{fig/H1_PE_NORM_HV1100v3.pdf}
   \label{H1PENorm}}
  \end{minipage}
%\hfill%%%
  \begin{minipage}{0.8\textwidth}
      \subfigure[光量補正後:Mean=0.9167, RMS=0.1174]{
\includegraphics[bb=255 306 822 575, width=1\textwidth]{fig/H1_PE_NORM_CORR_HV1100.pdf}
%\includegraphics[bb=255 36 822 575, width=1\textwidth]{fig/H1_PE_NORM_CORR_HV1100v2.pdf}
%\includegraphics[bb=255 36 822 575, width=1\textwidth]{fig/H1_PE_NORM_CORR_HV1100v3.pdf}
       \label{H1PENormCorr}}
  \end{minipage}
    \caption[規格化された光電子数分布]{規格化された光電子数分布。光量補正を行うことによって、分布の幅(RMS)が0.1778 $\rightarrow$ 0.1174と良くなっている。}
  \label{H1PECorr}
\end{figure}

\figref{H1PECorr}は、印加電圧1100 Vで測定した全光電子増倍管の$R_{\pe}$の分布を示す。\figref{H1PENorm}に\tabref{PECorrectionFactor}を用いて場所による光量補正を行うと\figref{H1PENormCorr}になる。

光量補正を行うことによって、$R_{\pe}$は平均値は$1.22 \rightarrow 0.92$に変化し、標準偏差は$0.17 \rightarrow 0.12$と改善した。補正後の結果から、相対的量子効率のばらつきは13\%程度である。




\subsection{電流増幅率曲線}
光電子増倍管の印加電圧を1000 Vから1300 Vまで50V刻みで変化させながら、全部で154本の光電子増倍管に対して上記の測定を行った。
\figref{GainDistribution1}、\figref{GainDistribution2}はその測定で得られた、各印加電圧での全光電子増倍管の電流増幅率分布を示す。これらのプロットから印加電圧1100 V$\sim$1250 Vの範囲で平均して$1\sim2\times 10^{6}$程度の電流増幅率が得られることが分かった。



\begin{figure}[htbp]
\centering
\includegraphics[bb=543 441 677 574, width=0.5\textwidth]{fig/CalibGainCurveLog.pdf}
\caption[電流増幅率曲線のフィッティング]{電流増幅率曲線のフィッティング。横軸に印加電圧、縦軸に電流増幅率をとり、両対数グラフにプロットした。フィッティング関数に$y=Ax+B$を使用した。}
\label{GainCurve}
\end{figure}

また、各印加電圧で計算した電流増幅率、横軸を印加電圧、縦軸を電流増幅率とした両対数軸にプロットし、直線$y=Ax+B$でフィッティングを行った結果の一例を\figref{GainCurve}に示す。
このフィットから得た係数$A, B$および、前述した相対的量子効率を考慮した上での、各光電子増倍管ごとに必要とされる電流増幅率から、その光電子増倍管に最適な印加電圧を逆算してもとめる。

今回フィットがうまくできなかったり、測定がうまくできなかった光電子増倍管については今後再試験をする予定である。
\newpage

\begin{figure}[htbp]
\centering
  \begin{minipage}{0.47\textwidth}
    \subfigure[印加電圧1000 V:Mean = $5.8\times10^{5}$、RMS = $1.4\times10^{4}$]{
\includegraphics[bb=255 36 822 575, width=1\textwidth]{fig/H1_GAIN_HV1000.pdf}
   \label{H1Gain1000}}
  \end{minipage}
\hfill%%%
  \begin{minipage}{0.47\textwidth}
    \subfigure[印加電圧1040 V:Mean = $8.0\times10^{5}$、RMS = $2.6\times10^{5}$]{
    \includegraphics[bb=255 36 822 575, width=1\textwidth]{fig/H1_GAIN_HV1050.pdf}
   \label{H1Gain1050}}
  \end{minipage}
  \hfill%%%
  \begin{minipage}{0.47\textwidth}
    \subfigure[印加電圧1100 V:Mean = $1.7\times10^{6}$、RMS = $2.9\times10^{5}$]{
    \includegraphics[bb=255 36 822 575, width=1\textwidth]{fig/H1_GAIN_HV1100.pdf}
   \label{H1Gain1100}}
  \end{minipage}
  \hfill%%%
  \begin{minipage}{0.47\textwidth}
    \subfigure[印加電圧1200 V:Mean = $1.8\times10^{6}$、RMS = $4.1\times10^{5}$]{
    \includegraphics[bb=255 36 822 575, width=1\textwidth]{fig/H1_GAIN_HV1200.pdf}
   \label{H1Gain1200}}
  \end{minipage}
     \caption[印加電圧別の電流増幅率分布1]{印加電圧別の電流増幅率分布1}
  \label{GainDistribution1}
\end{figure}


\begin{figure}[htbp]
\centering
 \begin{minipage}{0.47\textwidth}
    \subfigure[印加電圧1250 V:Mean = $2.3\times10^{6}$、RMS = $5.4\times10^{5}$]{
    \includegraphics[bb=255 36 822 575, width=1\textwidth]{fig/H1_GAIN_HV1250.pdf}
   \label{H1Gain1250}}
  \end{minipage}
\hfill
  \begin{minipage}{0.47\textwidth}
    \subfigure[印加電圧1150 V:Mean = $1.4\times10^{6}$、RMS = $3.1\times10^{5}$]{
    \includegraphics[bb=255 36 822 575, width=1\textwidth]{fig/H1_GAIN_HV1150.pdf}
   \label{H1Gain1150}}
  \end{minipage}
\hfill
    \begin{minipage}{0.47\textwidth}
    \subfigure[印加電圧1300 V:Mean = $2.9\times10^{6}$、RMS = $7.4\times10^{5}$]{
\includegraphics[bb=255 36 822 575, width=1\textwidth]{fig/H1_GAIN_HV1300.pdf}
   \label{H1Gain1300}}
  \end{minipage}
    \caption[印加電圧別の電流増幅率分布2]{印加電圧別の電流増幅率分布2}
  \label{GainDistribution2}
\end{figure}


\if0

\hfill%%%


  \begin{minipage}{0.47\textwidth}
    \subfigure[印加電圧1350 V:Mean=, RMS=]{\includegraphics[bb=255 36 822 575, width=1\textwidth]{fig/H1_GAIN_HV1350.pdf}
   \label{H1Gain1350}}
  \end{minipage}
  \hfill%%%
  \begin{minipage}{0.47\textwidth}
    \subfigure[印加電圧1400 V:Mean=, RMS=]{\includegraphics[bb=255 36 822 575, width=1\textwidth]{fig/H1_GAIN_HV1400.pdf}
   \label{H1Gain1400}}
  \end{minipage}
\fi%%%%%%%%%%%%%%%%%%%%%%%%%%%%%%%%%%%%%


\newpage
\subsection{印加電圧と光量の関係の問題点}

\figref{AppVoltagePE}はラン20の各CHに対して、横軸を印加電圧、縦軸を光量としてプロットした図である。光量モニター用の光電子増倍管(図左上)は一定の光量を観測しているにも関わらず、その他の光電子増倍管では、印加電圧を大きくすると光量が下がる傾向があるように見える。

原因の特定はできておらず、現在まだスタディ中の項目である。

\begin{figure}[htbp]
\centering
\includegraphics[bb=255 306 822 575, width=1\textwidth]{fig/RUN24_HV_PE_ZOOM.pdf}
\caption[印加電圧と光量の関係]{印加電圧と光量の関係}
\label{AppVoltagePE}
\end{figure}



本測定の測定原理のところに書いたように、今回は\equref{adcrms}のように測定したADC分布のMeanとRMSから、入射光電子数を計算する。印加電圧が大きい場所で、このMeanとRMSの線型性が違っていたら、光量が変わってくるため、LEDの光量を少なくして測定することを検討している。





%%%%%%%%%%%%%%%%%%%%%%%%%%%%%%%%%%%%%%%%%%%%%%%%%%%%%%%%%%%%%%%%%%%%%%%%%%%%%%%%
%%%%%%%%%%%%%%%%%%%%%%%%%%%%%%%%%%%%%%%%%%%%%%%%%%%%%%%%%%%%%%%%%%%%%%%%%%%%%%%%
\if0
\chapter{データ収集システム:MizuDAQ}

\section{概要}
\section{フロントエンド部}
\subsection{ATM}
\subsection{GONG}
\section{リアエンド部}
\subsection{SMP}
\section{トリガー部}
\subsection{TRG}
\fi

%%%%%%%%%%%%%%%%%%%%%%%%%%%%%%%%%%%%%%%%%%%%%%%%%%%%%%%%%%%%%%%%%%%%%%%%%%%%%%%%
%%%%%%%%%%%%%%%%%%%%%%%%%%%%%%%%%%%%%%%%%%%%%%%%%%%%%%%%%%%%%%%%%%%%%%%%%%%%%%%%
%\chapter{ニュートリノビーム測定}

%%%%%%%%%%%%%%%%%%%%%%%%%%%%%%%%%%%%%%%%%%%%%%%%%%%%%%%%%%%%%%%%%%%%%%%%%%%%%%%%
%%%%%%%%%%%%%%%%%%%%%%%%%%%%%%%%%%%%%%%%%%%%%%%%%%%%%%%%%%%%%%%%%%%%%%%%%%%%%%%%
\chapter{まとめ}

本研究では、T2K実験前置検出器ホールにて開発中の小型水チェレンコフ検出器``Mizuche''の開発を行った。検出器シミュレーションによる期待される性能評価と、強度解析・耐震解析の結果を踏まえた実機の構造の決定ならびに製作、そして、使用する光電子増倍管の相対的量子効率・電流増幅率の測定を行った。

MIzuche検出器は、水とニュートリノの反応によって生じる荷電粒子(主にミューオン)が放出するチェレンコフ光を、周りに設置した光電子増倍管で検出することによって、ニュートリノを観測する水チェレンコフ光検出器である。

T2K実験の後置検出器であるスーパーカミオカンデと同じニュートリノ反応標的(水)と検出原理(チェレンコフ光)を持つ本検出器により振動前のニュートリノ反応数を測定することにより、スーパーカミオカンデへの外挿を系統誤差を小さく抑えて行うことそ最終目標としている。

本検出器は有効体積0.5トンの内タンク(FV)と一回り大きい外タンク(OV)を同軸上に配置した2層構造をしている。外タンクと内タンクの間には300 mmのバッファー層を設け、FVの端で起こったニュートリノ反応によるミューオンでも十分なチェレンコフ光を発生させることができるようになっている。FVとOVは物理的に区切られており、その内部の水はそれぞれ独立して充填することが可能である。そのため、FVの水だけを抜き差しして測定を行うことが可能である。

本検出器の測定原理は、FV水ありと、FV水なしの2状態で測定を行い、その残差をFVで起きたニュートリノ反応数として計数することである。
これは2状態の差をとることにより、OVでの反応は相殺し、FVでの反応だけが残るからである。だたし、この測定原理が成り立つためには、2状態のOVでのニュートリノ反応検出効率が一致している必要がある。

そこで、ニュートリノ反応に対する検出器シミュレーションを行い、2状態のOVの検出効率をそれぞれ見積もった。
その結果、期待される総光量に対するカットを150 p.e.に設定すると、2状態のOVの検出効率はよく一致し、測定原理が成り立つことを示すことができた。またその際に、シグナルに対するOV混入イベントの割合を3\%程度にまで抑えて測定することが可能だということが分かった。

次に強度・耐震性の確認を行いながら、検出器の詳細設計を行った。強度・耐震性の解析にはANSYSという強度解析ツールを使用した。材料の引張強度に対して安全係数を3に設定した設計を行い、業者に製作を依頼した。検出器を満水試験による変形量は、ANSYSで得られた結果とほぼ同じ結果を示した。強度の安全性は問題ないと判断し、検出器を前置検出器ホール地下2階へインストールした。

光電子増倍管のキャリブレーションに関しては、必要数164本に対して、現在までに153本の相対的量子効率、電流増幅率曲線の測定を終えた。本測定のセットアップの再現性による光量補正を行った結果、相対的量子効率は13\%程度のばらつきがあることが分かった。今後は、合わせて測定した電流増幅率曲線を利用して、電流増幅率の制御を行い、ある入射光量に対して全ての光電子増倍管からの出力が一様となるよう印加電圧の調整を行う予定である。


今後の予定として、まず、検出器への光電子増倍管取り付けやケーブリングなどのアセンブリー作業を完了させる。次に、LED光源を用いた検出器応答のキャリブレーションや、ニュートリノビーム由来の壁からのミューオンを用いた光量キャリブレーションなどの作業を行う。そして、FV水ありの状態でニュートリノ反応数の測定を開始し、十分なデータを取得できた後は、水なしの状態での測定を開始する。そして、T2K実験の大強度ビームに対する小型水チェレンコフ検出器の実用性の検証、および前置検出器部分でのニュートリノ反応数測定精度2\%を目指したい。


%%%%%%%%%%%%%%%%%%%%%%%%%%%%%%%%%%%%%%%%%%%%%%%%%%%%
%%%%%%%%%%%%%%%%%%%%%%%%%%%%%%%%%%%%%%%%%%%%%%%%%%%%
\chapter*{謝辞}

修士課程の2年間は、とても短く感じられましたが、多くの方々に支えられて、充実した日々を送ることができました。ここに感謝の意を表したいと思います。本当にありがとうございます。

中家剛教授、市川温子准教授、そして小林隆教授には、T2K実験という世界最先端の実験場で本研究の機会を与えてくださったことに感謝いたします。

特に、市川温子准教授には、のんびりな私を要所要所で引き締めていただき、根気よく丁寧にここまでご指導いただいたことを深く感謝しております。

坂下健助教には、本研究を進める上でとてもたくさんのご助言をいただきました。
また、研究者としての姿勢を間近で学ばせていただきました。
村上明さんには、検出器シミュレーションをはじめ、本研究の至る所でたくさんお世話になりました。
山内隆寛くんに、年末の光電子増倍管の測定を手伝っていただいておかげで、測定をほぼ終わらせることができました。本当にありがとう。

検出器の作製に関して、スズノ技研株式会社の皆様に本当に感謝しております。できあがった検出器を初めて見たときの嬉しさはいまでも忘れません。本当にありがとうございます。

検出器のインストールや地下での作業には第一鉄工株式会社の皆様に大変お世話になりました。
いろいろと急なお願いが多いにも関わらず、引き受けてくださり本当にありがとうございます。

メカサポートの田井野ご夫妻と石井さんには、光電子増倍管の準備およびコネクタ付けの際には大変お世話になりました。本当にありがとうございます。

%この研究をみんなと一緒にできて本当に良かったと思います。

J-PARCでの生活では、大谷将士さん、木河達也くん、鈴木研人くん、矢野孝臣さん、とたまに来る松村知恵さんのおかげで、楽しく過ごせました。大谷さんは、何かと研究の進捗を気にかけて下さり、アドバイスなどもたくさんしていただきました。ありがとうございます。\newline

最後に、これまで研究生活を支えてくれた家族と最愛の人に心から感謝します。本当にありがとう。\newline
\begin{flushright}
2011年1月 吉日\\
%\UTF{9AD9}橋 将太
髙橋将太
\end{flushright}



%\input{bibbib}
%\bibliography{mt_ref}

\end{document}
