%%%%%%%%%%%%%%%%%%%%%%%%%%%%%%%%%%%%%%%%%%%%%%%%%%%%%%%%%%%%%%%%%%%%%%%%%%%%%%%%
%%%%%%%%%%%%%%%%%%%%%%%%%%%%%%%%%%%%%%%%%%%%%%%%%%%%%%%%%%%%%%%%%%%%%%%%%%%%%%%%
%\chapter{ニュートリノビーム測定}

%%%%%%%%%%%%%%%%%%%%%%%%%%%%%%%%%%%%%%%%%%%%%%%%%%%%%%%%%%%%%%%%%%%%%%%%%%%%%%%%
%%%%%%%%%%%%%%%%%%%%%%%%%%%%%%%%%%%%%%%%%%%%%%%%%%%%%%%%%%%%%%%%%%%%%%%%%%%%%%%%
\chapter{まとめ}

本研究では、T2K実験前置検出器ホールにて開発中の小型水チェレンコフ検出器``Mizuche''の開発を行った。検出器シミュレーションによる期待される性能評価と、強度解析・耐震解析の結果を踏まえた実機の構造の決定ならびに製作、そして、使用する光電子増倍管の相対的量子効率・電流増幅率の測定を行った。

MIzuche検出器は、水とニュートリノの反応によって生じる荷電粒子(主にミューオン)が放出するチェレンコフ光を、周りに設置した光電子増倍管で検出することによって、ニュートリノを観測する水チェレンコフ光検出器である。

T2K実験の後置検出器であるスーパーカミオカンデと同じニュートリノ反応標的(水)と検出原理(チェレンコフ光)を持つ本検出器により振動前のニュートリノ反応数を測定することにより、スーパーカミオカンデへの外挿を系統誤差を小さく抑えて行うことそ最終目標としている。

本検出器は有効体積0.5トンの内タンク(FV)と一回り大きい外タンク(OV)を同軸上に配置した2層構造をしている。外タンクと内タンクの間には300 mmのバッファー層を設け、FVの端で起こったニュートリノ反応によるミューオンでも十分なチェレンコフ光を発生させることができるようになっている。FVとOVは物理的に区切られており、その内部の水はそれぞれ独立して充填することが可能である。そのため、FVの水だけを抜き差しして測定を行うことが可能である。

本検出器の測定原理は、FV水ありと、FV水なしの2状態で測定を行い、その残差をFVで起きたニュートリノ反応数として計数することである。
これは2状態の差をとることにより、OVでの反応は相殺し、FVでの反応だけが残るからである。だたし、この測定原理が成り立つためには、2状態のOVでのニュートリノ反応検出効率が一致している必要がある。

そこで、ニュートリノ反応に対する検出器シミュレーションを行い、2状態のOVの検出効率をそれぞれ見積もった。
その結果、期待される総光量に対するカットを150 p.e.に設定すると、2状態のOVの検出効率はよく一致し、測定原理が成り立つことを示すことができた。またその際に、シグナルに対するOV混入イベントの割合を3\%程度にまで抑えて測定することが可能だということが分かった。

次に強度・耐震性の確認を行いながら、検出器の詳細設計を行った。強度・耐震性の解析にはANSYSという強度解析ツールを使用した。材料の引張強度に対して安全係数を3に設定した設計を行い、業者に製作を依頼した。検出器を満水試験による変形量は、ANSYSで得られた結果とほぼ同じ結果を示した。強度の安全性は問題ないと判断し、検出器を前置検出器ホール地下2階へインストールした。

光電子増倍管のキャリブレーションに関しては、必要数164本に対して、現在までに153本の相対的量子効率、電流増幅率曲線の測定を終えた。本測定のセットアップの再現性による光量補正を行った結果、相対的量子効率は13\%程度のばらつきがあることが分かった。今後は、合わせて測定した電流増幅率曲線を利用して、電流増幅率の制御を行い、ある入射光量に対して全ての光電子増倍管からの出力が一様となるよう印加電圧の調整を行う予定である。


今後の予定として、まず、検出器への光電子増倍管取り付けやケーブリングなどのアセンブリー作業を完了させる。次に、LED光源を用いた検出器応答のキャリブレーションや、ニュートリノビーム由来の壁からのミューオンを用いた光量キャリブレーションなどの作業を行う。そして、FV水ありの状態でニュートリノ反応数の測定を開始し、十分なデータを取得できた後は、水なしの状態での測定を開始する。そして、T2K実験の大強度ビームに対する小型水チェレンコフ検出器の実用性の検証、および前置検出器部分でのニュートリノ反応数測定精度2\%を目指したい。