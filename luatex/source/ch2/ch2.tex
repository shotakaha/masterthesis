\chapter{T2K長基線ニュートリノ振動実験}

%%%%%%%%%% %%%%%%%%%% %%%%%%%%%% %%%%%%%%%% %%%%%%%%%% %%%%%%%%%%
\section{実験概要・目的}
T2K(Tokai to Kamioka)長基線ニュートリノ振動実験の概念図を\figref{T2KOverview}に示す。

T2K実験\cite{jhfnu}は茨城県東海村にあるJ-PARC大強度陽子加速器施設で
生成したミューオンニュートリノビームを岐阜県飛騨市のスーパーカミオカンデ検出器で
観測する全長295 kmの長基線ニュートリノ振動実験である。
本実験は2009年4月に稼働開始した。
T2K実験では世界最大強度のニュートリノビームと世界最大の水チェレンコフ検出器スーパーカミオカンデを用いて、

\begin{enumerate}
\item ミューオンニュートリノ消失による混合角$\theta_{23}$および$\Delta m_{23}$の精密測定
\item 電子ニュートリノ出現モードによる混合角$\theta_{13}$の世界初観測
\end{enumerate}

を世界最高感度で実現することを目標としている。

\begin{figure}[htbp]
\centering
%\includegraphics[bb=54 392 539 489, width=1\textwidth]{fig/T2KOverview.pdf}
\includegraphics[bb=131 315 483 418, width=1\textwidth]{fig/T2KOverview2.pdf}
\caption[T2K実験の概要]{T2K実験の概要。茨城県東海村のJ-PARC加速器施設で生成した人工ミューオンニュートリノを、295km離れたスーパーカミオカンデで観測し、ニュートリノ振動測定を行う。}
\label{T2KOverview}
\end{figure}

\subsection{ニュートリノ振動解析}

生成点直後と長距離飛行後のニュートリノの状態をそれぞれ前置検出器、後置検出器で測定を行う。
前置検出器での測定結果を外挿して、後置検出器の結果を予測し、
その値を後置検出器の測定結果と比較することにより、ニュートリノ振動解析を行う。
このとき、振動確率を表す\equref{nuchange}より、混合角は主にニュートリノ反応数の増減から、
質量二乗差は主にエネルギースペクトルの歪みから求められる。

T2K実験では、ニュートリノ生成点から280m下流に配置した前置検出器と、295 km離れた後置検出器にスーパーカミオカンデを使用する。
前置検出器での結果$N_{\nd}^{obs}$を外挿して、
スーパーカミオカンデでのニュートリノ反応数予測$N_{\sk}^{exp}$を求める式は次のようになる。

\begin{equation}
N_{\sk}^{exp}  =  R_{Far/Near} \times N_{\nd}^{obs}
\label{Extrapolation}
\end{equation}

ここで$R_{Far/Near}$はF/N比(Far-to-Near ratio)と呼ばれるもので、
モンテカルロシミュレーションにより求めた前置検出器、スーパーカミオカンデの、
それぞれのニュートリノ反応数$N_{\nd}^{MC}, N_{\sk}^{MC}$を用いて次式で定義される数である。

\begin{equation}
R_{Far/Near} \equiv \frac{N_{\sk}^{MC}}{N_{\nd}^{MC}} = \frac{\int \Phi_{\sk}^{MC} \times \sigma \times \epsilon_{\sk}}{\int \Phi_{\nd}^{MC} \times \sigma \times \epsilon_{\nd}}
\label{FN}
\end{equation}

ここで、右辺の各変数は以下の通りである。
\begin{description}
\item [$\blacksquare\ \Phi_{\sk, \nd}^{MC}$] $\cdots$ MCによるスーパーカミオカンデ、前置検出器でのエネルギースペクトル
\item [$\blacksquare\ \sigma$] $\cdots$ ニュートリノ反応断面積
\item [$\blacksquare\ \epsilon_{\sk, \nd}$] $\cdots$ スーパーカミオカンデ、前置検出器の検出効率
\end{description}

\equref{FN}より前置検出器のエネルギースペクトル、反応標的、検出効率などをスーパーカミオカンデのそれに近づけることで、それらに付いてくる不定性がお互い打ち消しあい、F/N比の系統誤差を小さくすることができる。その結果、スーパーカミオカンデでのニュートリノ反応数予測の精度を向上させることができる。



%%%%%%%%%% %%%%%%%%%% %%%%%%%%%% %%%%%%%%%% %%%%%%%%%% %%%%%%%%%%
\section{J-PARC加速器およびニュートリノビームライン}

\subsection{J-PARC加速器}

J-PARC加速器の構成を\figref{JPARC}に示す。
全長330 m の線形加速器リニアック(LINAC)で加速された陽子は、
全周350 m の 3 GeV陽子シンクロトロン(RCS)、
全周1600 m の陽子シンクロトロン(MR)の順に加速され、
最終的にビームエネルギー30G eV、ビーム強度750 kW (デザイン値)にまで到達する。
その後、超伝導磁石を用いた速い取り出し(FX)によって2 $\sim$ 4秒の間隔でニュートリノビームラインへと蹴り出される。
1スピルあたり8バンチ、1バンチの幅58 nsec、バンチ間隔581 nsecのビーム構造をしている。
これらのJ-PARC加速器の陽子ビームパラメータを\tabref{JPARCBeamlineSpec}にまとめた。
なお、ビームエネルギー30 GeV、ビーム強度750 kWを達成するために、今後これらのパラメータを変更する可能性もある。

\begin{figure}[htbp]
\centering
\includegraphics[bb=0 0 432 274, width=0.7\textwidth]{fig/T2KJPARCBL.jpg}
\caption[J-PARC加速器の構成]{J-PARC加速器の構成。LINAC、RCS、MRで徐々に加速された陽子は最終的に30 GeVのエネルギーに達する。(図はJ-PARC公式HPより)}
\label{JPARC}
\end{figure}

\begin{table}[htbp]
\caption[J-PARC加速器ビームパラメータのデザイン値]{J-PARC加速器ビームパラメータのデザイン値}
\begin{center}
\begin{tabular}{ccl}
\hline \hline
ビームエネルギー & 30 & [GeV]\\
ビーム強度 & 750 &[kW] (現在は約100 kW)\\
1スピル当たりの陽子数 & $3.3 \times 10^{14}$ & [pps] (現在は$7\times10^{7}$ pps)\\
スピル周期 & 2.11 & [sec] (現在は3.2 sec)\\
スピル構造 & 8 &[bunches/spill]\\
バンチ間隔 & 581 & [nsec]\\
バンチ幅 & 58 & [nsec] (現在は半値幅10 nsec)\\
\hline \hline
\multicolumn{3}{r}{pps = protons/spill}\\
\end{tabular}
\end{center}
\label{JPARCBeamlineSpec}
\end{table}%

\newpage
\subsection{ニュートリノビームライン}
ニュートリノビームラインの構成を\figref{T2KNeutrinoBeamline}に示す。MRで30GeVまで加速された陽子は、超電導磁石によって曲げられ、ニュートリノビームラインに導かれる。その後、陽子ビームは炭素標的に衝突し$\pi$中間子、K中間子を生成する。これらの荷電粒子を電磁ホーンによって収束させてから、崩壊トンネルに入射させる。崩壊トンネル内では、粒子の崩壊によってニュートリノやその他の粒子が生成する。ニュートリノ以外の粒子はビームダンプによって堰き止められるため、ニュートリノのみが、前置検出器群よびスーパーカミオカンデに向かって飛んでいくことができる。

\begin{figure}[htbp]
\centering
\includegraphics[bb=0 0 970 208, width=1\textwidth]{fig/T2KNBL2.pdf}
\caption[ニュートリノビームラインの構成]{ニュートリノビームラインの構成。MRで30 GeVまで加速された陽子は、超電導磁石によって曲げられたのち、炭素標的に衝突し荷電粒子を生成する。荷電粒子は電磁ホーンによって収束させられ、崩壊トンネル中でニュートリノへと崩壊し、前置検出器群・スーパーカミオカンデへと飛んでいく。}
\label{T2KNeutrinoBeamline}
\end{figure}

%\paragraph{炭素標的}
%\sout{T2Kニュートリノビームラインでは炭素をニュートリノ生成標的として用いている。陽子ビームと炭素の散乱で、大量の$\pi$中間子やK中間子が生成する。この$\pi$中間子やK中間子の崩壊反応を利用してニュートリノを生成する}

%\paragraph{電磁ホーン}
%\sout{大強度のニュートリノビームを作るためには、電磁ホーンという装置を使ってニュートリノビームを収束させる必要がある。T2Kビームラインには合計3つの電磁ホーンが建設されており、そえぞれに最大320kAの大電流を流すことができる。(ただし、現段階は250kAで運転)
%
%第1ホーン内部には炭素標的があり、陽子ビームが衝突して二次粒子(主に荷電パイオンと荷電ケーオン)を生成すると同時に、それらを収束させる。第2、第3ホーンはさらに生成粒子を収束させる。}

%\paragraph{崩壊空洞}
%\sout{Decay Volume。ニュートリノ生成点(炭素標的のある位置)から下流約100mには、内部を真空に引かれた空洞を用意してある。この空洞内で、$\pi$中間子は、ミューオンとミューオンニュートリノに2体崩壊する。このときのニュートリノを集めてビームとする。}


\if0  %%%%%%%%%% %%%%%%%%%%
\subsection{\sout{MUMON}}

\begin{figure}[htb]
\begin{center}
\includegraphics[bb=0 0 319 287, scale=0.5]{fig/temp.pdf}
\caption[MUMON]{MUMON}
\label{MUMON}
\end{center}
\end{figure}

\fi %%%%%%%%%% %%%%%%%%%%

%%%%%%%%%% %%%%%%%%%% %%%%%%%%%% %%%%%%%%%% %%%%%%%%%% %%%%%%%%%%
\section{T2K前置検出器群}
\subsection{INGRID}
INGRIDはニュートリノビーム軸上に置かれた検出器である。合計14個のモジュールからなる大質量の検出器である。ニュートリノビームの軸中心がどの方向を向いているのかを毎日確認することができる。


\subsection{オフアクシス検出器}
オフアクシス検出器はニュートリノ生成点から下流280 mに設置されており、ビーム軸からずれたスーパーカミオカンデ方向を向いている。
その断面図を\figref{TOAD}に示す。オフアクシス検出器はP0D、TPC、FGD、ECAL、SMRDの5つの検出器で構成される複合型検出器である。飛跡検出をメインに、運動量再構成、エネルギー再構成、粒子識別を行い、振動前のニュートリノのフラックスおよびエネルギーの測定を行う。測定結果を基にスーパーカミオカンデでのフラックスおよびエネルギースペクトルを予測する。

\begin{figure}[htbp]
\centering
%\includegraphics[bb=189 528 418 758, width=0.8\textwidth]{fig/T2KOAD.pdf}
\includegraphics[bb=0 0 873 773, width=0.7\textwidth]{fig/T2KND280.png}
%\includegraphics[bb=0 0 1213 1073, width=0.8\textwidth]{fig/T2KND280.pdf}
%\includegraphics[bb=14 14 4864 4305, clip, width=0.8\textwidth]{fig/T2KND280.eps}
\caption[T2Kオフアクシス検出器]{T2Kオフアクシス検出器}
\label{TOAD}
\end{figure}

\if0%%%%%%%%%%%%%%%%%%%%%%%%%%%%%%%%%%%%%
\paragraph{P0D(Pi-zero Detector)}
\sout{P0Dはプラスチックシンチレータと水標的で構成される検出器である。電磁石内の上流に設置されている。中性カレント反応によって生じた$\pi^{0}$粒子を測定することができる。
水と鉛が入っている?
%
$\pi^{0} \rightarrow 2 \gamma$の崩壊によって生じた$\gamma$を検出する。}

\paragraph{FGD(Fine Grained Detector)}
\sout{エフジーディー。Fine Grained Detectorの略。}

\paragraph{TPC(Time Projection Chamber)}
\sout{ティーピーシー。Time Projection Chamberの略。}

\paragraph{ECAL(Electromagnetic CALorimeter)}
\sout{ECALはプラスチックシンチレーターと鉛フィルムから構成される検出器で、P0D、FGD、TPCの周りを囲むように設置されている。
$\pi^{0}$中間子の崩壊によって生じる$\gamma$を検出する。また電子ニュートリノによる荷電カレント反応によって生じた電子も検出できる。}

\paragraph{SMRD(Side Muon Range Detector)}
\sout{電磁石の鉄ヨークの隙間にプラスティックシンチレータが挟んである。ニュートリノ反応によるミューオンで、横方向にすり抜けてしまったものを検出する。}

\paragraph{電磁石}
\sout{T2K電磁石にはCERNのUA1実験のものを使用。約2900Aの電流を流し、0.2Tの磁場をかけ、ニュートリノ反応によって生じた荷電粒子の運動量を測定する。電磁石内部の大きさは3.5m$\times$3.6m$\times$7.0m}

\comment{オフアクシス検出器について簡単にまとめる。SKとは違うんだよ、ということを書く}
\fi


\if0 %%%%%%%%%% %%%%%%%%%%
\subsection{\sout{INGRID}}
ビーム軸に沿って設置。
10m x 10mの大きさの十字型。
ビーム中心がずれていないか測定。

\begin{figure}[htb]
\centering
\includegraphics[bb=0 0 319 287, scale=0.5]{fig/temp.pdf}
\caption[INGRID]{INGRID}
\label{INGRID}
\end{figure}

\fi %%%%%%%%%% %%%%%%%%%%


\newpage
%%%%%%%%%% %%%%%%%%%% %%%%%%%%%% %%%%%%%%%% %%%%%%%%%% %%%%%%%%%%
\section{後置検出器:スーパーカミオカンデ}

T2K実験ではスーパーカミオカンデを、ニュートリノ発生点から295kmの地点に置かれた後置検出器として使用する。

スーパーカミオカンデは、岐阜県飛騨市神岡町の神岡鉱山茂住坑内に、東京大学宇宙線研究所付属の観測装置として建設された、水チェレンコフ光検出器である。宇宙線起源のミューオンによるバックグラウンドを減らすため、池の山山頂の地下1000m(2700m w.q.e)に建設された。実際に検出器付近での宇宙線ミューオンの強度は、地表での強度の約$10^{-5}$となっており、スーパーカミオカンデにおける宇宙線ミューオン事象の頻度は4Hzにまで抑えられている。

スーパーカミオカンデ検出器の全体図を\figref{SuperKamiokande}に示す。スーパーカミオカンデ検出器の本体となるタンクは、直径39.3m、高さ41.4mの円筒形をしており、その中は総質量5万トンの超純水で満たされている。タンクの内部は光学的に内タンク(直径33.8m、高さ36.2m、有効体積22.5トン)と外タンクに分けられており、内タンクには直径20インチの光電子増倍管約11200本が内向きに、外タンクには直径8インチの光電子増倍管約1900本が外向きに、それぞれ取り付けられている。

内タンクは粒子検出の主となる部分であり、タンクの中もしくは外で起こった反応により生じた荷電粒子が、水中を通過する際に放出するチェレンコフ光を、内タンク壁面に並べられた光電子増倍管で検出し、その光量・到達時間・リングパターンなどから、粒子の種類・エネルギー・発生点・運動方向などを決定する。

外タンクは、岩盤からの$\gamma$線や中性子によるバックグラウンド事象の除去および外部から入射する粒子(主に宇宙線ミューオン)や外部に抜ける粒子の識別、のために利用されている。

\begin{figure}[!htb]
\centering
\includegraphics[bb=86 439 501 740, width=1\textwidth]{fig/T2KSK.pdf}
\caption[スーパーカミオカンデ]{スーパーカミオカンデ}
\label{SuperKamiokande}
\end{figure}

現在までにスーパーカミオカンデは、そのずば抜けた性能により、太陽ニュートリノ・大気ニュートリノなどの自然から来るニュートリノを観測し、ニュートリノの質量に関する多くの情報をもたらしている

\if0
\section{T2K実験の現状}
余力があればNsk/Nndの最近の結果(INGRID, Off-Axis)に触れて、cancellationをdemonstrateする

\begin{quote}
余力があればNsk/Nndの最近の結果(INGRID, Off-Axis)に触れて、cancellationをdemonstrateする
\end{quote}

余力があればNsk/Nndの最近の結果(INGRID, Off-Axis)に触れて、cancellationをdemonstrateする
\fi